
\chapter{Oración Inicial}

\gresetinitiallines1

\gregorioscore{gabc/an--just-domine-labia-mea}

\gregorioscore{deusinadj}

\chapter{Invitatorio}


%\gregorioscore{gabc/invitatory-avemaria}

%\gregorioscore{gabc/va--venite_exsultemus--hartker}

\section{Tono Simple}

\rubrics{El cantor entona esta antífona:}

\gregorioscore{gabc/invitatory-avemaria-init}

\tr{Dios te salve, María, llena eres de gracia, el Señor es contigo.}

\rubrics{El coro repite esta antífona.}

\rubrics{El cantor canta los versículos y el coro canta las antífonas entre los versículos, alternándose el canto de la antífona completa con el canto de la segunda parte solamente, \emph{Dóminus tecum}.}

\subsection{Salmo 94.}

\gregorioscore{gabc/94-7a-1}

\tr{Venid, regocijémonos en el Señor; cantemos con júbilo las alabanzas de Dios, Salvador nuestro. Corramos a presentarnos ante su acatamiento, dándole gracias, y entonándole himnos con júbilo.}

\gresetinitiallines0
\gregorioscore{gabc/invitatory-avemaria}

\tr{Dios te salve, María, llena eres de gracia, el Señor es contigo.}


\gregorioscore{gabc/94-7a-2}

\tr{Porque el Señor es el Dios grande, y un rey más grande que todos los dioses. Porque en su mano tiene toda la extensión de la tierra, y suyos son los más encumbrados montes.}

\gregorioscore{gabc/invitatory-dominustecum}

\gregorioscore{gabc/94-7a-3}

\tr{Suyo es el mar, y obra es de sus manos: y hechura de sus manos es la tierra. (de rodillas) Venid, pues, adorémosle, postrémonos: derramando lágrimas en la presencia del Señor que nos ha creado: Pues Él es el Señor Dios nuestro: y nosotros el pueblo a quien Él apacienta, y ovejas de su grey.}

\gregorioscore{gabc/invitatory-avemaria}

\gregorioscore{gabc/94-7a-4}

\tr{Hoy mismo, si oyereis su voz, guardaos de endurecer vuestros corazones, Como sucedió, dice el Señor, cuando provocaron mi ira, poniéndome a prueba en el desierto, en donde vuestros padres me tentaron, me probaron, y vieron mis obras.}

\gregorioscore{gabc/invitatory-dominustecum}

\gregorioscore{gabc/94-7a-5}

\tr{Por espacio de cuarenta años estuve irritado contra esta generación, y dije: Siempre está descarriado el corazón de este pueblo. Ellos no conocieron mis caminos; por lo que juré airado que no entrarían en mi reposo.}

\gregorioscore{gabc/invitatory-avemaria}

\gregorioscore{gabc/94-7a-6}

\tr{Gloria al Padre, al Hijo, * y al Espíritu Santo. Como era en el principio, ahora y siempre, * por los siglos de los siglos. Amén.}

\gregorioscore{gabc/invitatory-dominusavedominus}

\gresetinitiallines1

\chapter{Himno}

\subsection{Melodía del Oficio Parvo}

\gregorioscore{gabc/hy--quem_terra_sidera}


\chapter{Salmodia}

\section{Domingo, Lunes y Jueves}

\gregorioscore{gabc/benedictatu-init}

\subsection{Salmo 8}

\gresetinitiallines0
\gregorioscore{gabc/8-4A}

\setcounter{versecount}{1}
\startParallel
\input psalms/8-4A
\stopParallel

\gregorioscore{gabc/benedictatu}

\bigskip


\gresetinitiallines1
\gregorioscore{gabc/an--sicut_myrrha--nocturnale_2002-init}

\subsection{Salmo 18}

\gresetinitiallines0
\gregorioscore{psalms/18-4A}

\setcounter{versecount}{1}
\startParallel
\input psalms/18-4Avv
\stopParallel

\gregorioscore{gabc/an--sicut_myrrha--nocturnale_2002}

\bigskip


\gresetinitiallines1
\gregorioscore{gabc/an--ante_torum--nocturnale_2002-init}

\subsection{Salmo 23}

\gresetinitiallines0
\gregorioscore{psalms/23-4A}

\setcounter{versecount}{1}
\startParallel
\input psalms/23-4Avv
\stopParallel

\gregorioscore{gabc/an--ante_torum--nocturnale_2002}

\rubrics{Con esto finaliza la Salmodia --- debes ir ahora al Versículo \emph{Diffusa est gratia} page \pageref{diffusa}.}

\section{Martes y Viernes}

\gresetinitiallines1
\gregorioscore{gabc/an--specie_tua_(ant)--nocturnale_2002-init}

\subsection{Salmo 44}

\gresetinitiallines0
\gregorioscore{gabc/44-7c}

\setcounter{versecount}{1}
\startParallel
\input psalms/44-7c
\stopParallel

\setcounter{versecount}{0}

\gregorioscore{gabc/an--specie_tua_(ant)--nocturnale_2002}

\bigskip


\gresetinitiallines1
\gregorioscore{gabc/an--adjuvabit_eam--nocturnale_2002-init}


\subsection{Salmo 45}

\gresetinitiallines0
\gregorioscore{gabc/45-7c}

\startParallel
\setcounter{versecount}{1}
\input psalms/45-7c
\stopParallel

\gregorioscore{gabc/an--adjuvabit_eam--nocturnale_2002}

\bigskip


\gresetinitiallines1
\gregorioscore{gabc/an--sicut_laetantium--nocturnale_2002-init}

\subsection{Salmo 86}

\gresetinitiallines0
\gregorioscore{gabc/86-7c}

\startParallel
\setcounter{versecount}{1}
\input psalms/86-7c
\stopParallel



\gregorioscore{gabc/an--sicut_laetantium--nocturnale_2002}

\rubrics{Con esto finaliza la Salmodia --- debes ir ahora al Versículo \emph{Diffusa est gratia} page \pageref{diffusa}.}

\newpage

\section{Miércoles y Sábado}

\gresetinitiallines1
\gregorioscore{gabc/an--gaude_maria_virgo--nocturnale_2002-init}

\subsection{Salmo 95}

\gresetinitiallines0
\gregorioscore{gabc/95-4c}

\startParallel
\setcounter{versecount}{1}
\input psalms/95-4c
\stopParallel

\gregorioscore{gabc/an--gaude_maria_virgo--nocturnale_2002}

\bigskip

\gresetinitiallines1
\gregorioscore{gabc/an--dignare_me_laudare--nocturnale_2002}

\subsection{Salmo 96}

\gresetinitiallines0
\gregorioscore{gabc/96-4c}

\startParallel
\setcounter{versecount}{1}
\input psalms/96-4c
\stopParallel

\gregorioscore{gabc/an--dignare_me_laudare--nocturnale_2002}

\bigskip

\rubrics{Next is Salmo 97.
La mayor parte del tiempo la antífona es \emph{Post partum}, sin embargo, en el Tiempo de Adviento y en la fiesta de la Anunciación el 25 de marzo, la antífona es \emph{Angelus Domini}, page \pageref{noc3ps3adv}.}

\gresetinitiallines1
\gregorioscore{gabc/an--post_partum-init}

\subsection{Salmo 97}

\gresetinitiallines0
\gregorioscore{gabc/97-4Astar}

\startParallel
\setcounter{versecount}{1}
\input psalms/97-4Astar
\stopParallel

\gregorioscore{gabc/an--post_partum}

\bigskip

\label{noc3ps3adv}
\rubrics{Así es como este tercer Salmo del tercer nocturno se canta en Adviento y en la fiesta de la Anunciación:

\gresetinitiallines1
%\gregorioscore{gabc/an--angelus_domini_(ad_matutinum_in_adv._et_ann.tionis)--nocturnale_2002}

\gregorioscore{gabc/an--angelus_domini_nuntiavit--solesmes}

\subsection{Salmo 97}

\gresetinitiallines0
\gregorioscore{gabc/97-1f}

\startParallel
\setcounter{versecount}{2}
\input psalms/97-1f
\switchcolumn
\tr{Cantad al Señor un cántico nuevo, porque ha hecho maravillas:
Su diestra le ha dado la victoria, su santo brazo. 

El Señor da a conocer su victoria, revela a las naciones su justicia: 

Se acordó de su misericordia y su fidelidad en favor de la casa de Israel. 

Los confines de la tierra han contemplado * la victoria de nuestro Dios.

Aclamad al Señor, tierra entera; * gritad, vitoread, tocad:

Tañed la cítara para el Señor, suenen los instrumentos: * con clarines y al son de trompetas,

Aclamad al Rey y Señor. * Retumbe el mar y cuanto contiene, la tierra y cuantos la habitan;

Aplaudan los ríos, aclamen los montes al Señor, * que llega para regir la tierra.

Regirá el orbe con justicia * y los pueblos con rectitud.

Gloria al Padre...}
\stopParallel

\gregorioscore{gabc/an--angelus_domini_nuntiavit--solesmes}

%\gregorioscore{gabc/an--angelus_domini_(ad_matutinum_in_adv._et_ann.tionis)--nocturnale_2002}

\setcounter{versecount}{0}

\chapter{Versículos y Padre Nuestro}
\label{diffusa}

Matins continues with the following versicles:

\gregorioscore{gabc/diffusa-est}

%\Vbar Diffúsa est grátia in lábiis tuis.

%\Rbar Proptérea benedíxit te Deus in aetérnum.

%Pater noster (secreto usque ad)

%\Vbar Et ne nos indúcas in tentatiónem.

%\Rbar Sed líbera nos a malo.

(Absolution)

\startParallel
\latin{Précibus et méritis beátæ Maríæ semper Vírginis et ómnium Sanctórum, perdúcat nos Dóminus ad regna cælórum. Amen.}
\vern{By the prayers and merits of the blessed Mary ever Virgin, and of all the Saints, may the Lord bring us to the kingdom of heaven. Amen.}
\stopParallel

\chapter{Lessons and Responsories}

%The Responsories are the big things. Most days you just sing two Responsories and finish with the Te Deum. In that case you would sing the first \textbf{without} the \emph{Gloria Patri}, then sing the second \textbf{with} the \emph{Gloria Patri}. In penitential times, such as Septuagesima, Lent and Advent, the \emph{Te Deum} is not sung. In that case you would sing the first two Responsories \textbf{without} a \emph{Gloria Patri} then finish off the third Responsory \textbf{with} a \emph{Gloria Patri}.

\section{Outside Advent}

\rubrics{If a priest is present, then this next line addresses him with the words: \emph{Jube domne, benedícere.} Then the priest gives the \emph{Benedictio}. Otherwise, they are addressed to Our Lord, and the celebrant gives the \emph{Benedictio} on His behalf.}

\Vbar . Jube Dómine, benedícere.

\emph{Benedictio.} Nos cum prole pia benedícat Virgo María. Amen.

\subsection{Lectio 1, Sir 24:11-13}

\startParallel
In ómnibus réquiem quaesívi, et in hereditáte Dómini morábor.
Tunc praecépit, et dixit mihi Creátor ómnium: et qui creávit me, requiévit in tabernáculo meo.
Et dixit mihi: In Jacob inhábita, et in Israël hereditáre, et in eléctis meis mitte radíces.
\switchcolumn
\tr{In all these I sought rest, and I shall abide in the inheritance of the Lord. Then the creator of all things commanded, and said to me: and he that made me, rested in my tabernacle, and he said to me: Let thy dwelling be in Jacob, and thy inheritance in Israel, and take root in my elect.}
\switchcolumn
\Vbar. Tu autem, Dómine, miserére nobis.

\Rbar. Deo grátias.
\switchcolumn
\tr{Do thou, O Lord, have mercy on us.

Thanks be to God.}
\stopParallel

\subsection{First Responsory}

\gresetinitiallines1
\gregorioscore{gabc/re--sancta_et_immaculata--sandhofe}

\Vbar. Jube Dómine, benedícere.

\emph{Benedictio.} Ipsa Virgo Vírginum intercédat pro nobis ad Dóminum. Amen.

\subsection{Lectio 2, Sir 24:15-16}

\startParallel
Et sic in Sion firmáta sum, et in civitáte sanctificáta simíliter requiévi, et in Jerúsalem potéstas mea.
Et radicávi in pópulo honorificáto, et in parte Dei mei heréditas illíus, et in plenitúdine sanctórum deténtio mea.
\switchcolumn
\tr{And so was I established in Sion, and in the holy city likewise I rested, and my power was in Jerusalem. And I took root in an honourable people, and in the portion of my God his inheritance, and my abode is in the full assembly of saints.}
\switchcolumn
\Vbar. Tu autem, Dómine, miserére nobis.

\Rbar. Deo grátias.
\switchcolumn
\tr{Do thou, O Lord, have mercy on us.

Thanks be to God.}
\stopParallel

\subsection{Second Responsory}

\gregorioscore{gabc/re--beata_es_virgo_maria--baronius}

\rubrics{Outside Septuagesima/Lent,  sing this \emph{Gloria Patri} part. During Septuagesima, skip the next \emph{Gloria Patri} part and jump to \emph{Jube Domine/domne}.}

\gregorioscore{gabc/re--beata--gloriapatri}

\Vbar. Jube Dómine, benedícere.

\emph{Benedictio.} Per Vírginem matrem concédat nobis Dóminus salútem et pacem. Amen.

\subsection{Lectio 3, Sir 24:17-20}

\startParallel
Quasi cedrus exaltáta sum in Líbano, et quasi cypréssus in monte Sion:
Quasi palma exaltáta sum in Cades, et quasi plantátio rosae in Jéricho:
Quasi olíva speciósa in campis, et quasi plátanus exaltáta sum juxta aquam in platéis.
Sicut cinnamómum et bálsamum aromatízans odórem dedi; quasi myrrha elécta dedi suavitátem odóris.
\switchcolumn
\tr{I was exalted like a cedar in Libanus, and as a cypress tree on mount Sion. I was exalted like a palm tree in Cades, and as a rose plant in Jericho: As a fair olive tree in the plains, and as a plane tree by the water in the streets, was I exalted. I gave a sweet smell like cinnamon, and aromatical balm: I yielded a sweet odour like the best myrrh.}
\switchcolumn
\Vbar. Tu autem, Dómine, miserére nobis.

\Rbar. Deo grátias.
\switchcolumn
\tr{Do thou, O Lord, have mercy on us.

Thanks be to God.}
\stopParallel

\rubrics{Outside Septuagesima and Lent, or on Feasts of the Blessed Virgin, go to the \emph{Te Deum}, page \pageref{tedeum}. During Septuagesima and Lent, finish with the next Responsory, \emph{Felix namque}.}

\subsection{Third Responsory}

\gregorioscore{gabc/re--felix_namque--hartker}

\rubrics{Thus ends Matins. Go straight to Lauds. If you're not ready for Lauds, you can finish up the Unofficial Conclusion.}

\section{During Advent}

\Vbar. Jube Dómine, benedícere.

\emph{Benedictio.} Nos cum prole pia benedícat Virgo María. Amen.

\subsection{Lectio 1, Luc 1:26-28}

\startParallel
Missus est Angelus Gábriel a Deo in civitátem Galilaéae, cui nomen Názareth,
Ad Vírginem desponsátam viro, cui nomen erat Joseph, de domo David: et nomen Vírginis María.
Et ingréssus Angelus ad eam dixit: Ave grátia plena: Dóminus tecum: benedícta tu in muliéribus.
\switchcolumn
\tr{The angel Gabriel was sent from God into a city of Galilee, called Nazareth, To a virgin espoused to a man whose name was Joseph, of the house of David: and the virgin’s name was Mary. And the angel being come in, said unto her: Hail, full of grace, the Lord is with thee: blessed art thou among women.}
\switchcolumn
\Vbar. Tu autem, Dómine, miserére nobis.

\Rbar. Deo grátias.
\switchcolumn
\tr{Do thou, O Lord, have mercy on us.

Thanks be to God.}
\stopParallel


\subsection{First Responsory}

\gregorioscore{gabc/re--missus_est_gabriel--sandhofe}

\Vbar. Jube Dómine, benedícere.

\emph{Benedictio.} Ipsa Virgo Vírginum intercédat pro nobis ad Dóminum. Amen.

\subsection{Lectio 2, Luc 1:29-33}

\startParallel
Quae cum audisset, turbata est in sermone ejus, et cogitabat qualis esset ista salutatio.
Et ait Angelus ei: Ne timeas, Maria: invenisti enim gratiam apud Deum:
Ecce concipies in utero, et paries filium, et vocabis nomen ejus Jesum:
Hic erit magnus, et Filius Altissimi vocabitur, et dabit illi Dominus Deus sedem David patris ejus: et regnabit in domo Iacob in aeternum,
Et regni ejus non erit finis.
\switchcolumn
\tr{Who having heard, was troubled at his saying and thought with herself what manner of salutation this should be. And the angel said to her: Fear not, Mary, for thou hast found grace with God. Behold thou shalt conceive in thy womb and shalt bring forth a son: and thou shalt call his name Jesus.He shall be great and shall be called the Son of the Most High. And the Lord God shall give unto him the throne of David his father: and he shall reign in the house of Jacob for ever. And of his kingdom there shall be no end.}
\switchcolumn
\Vbar. Tu autem, Dómine, miserére nobis.

\Rbar. Deo grátias.
\switchcolumn
\tr{Do thou, O Lord, have mercy on us.

Thanks be to God.}
\stopParallel

\subsection{Second Responsory}

\gregorioscore{gabc/re--ave_maria--sandhofe}

\bigskip

\rubrics{For Feasts of Our Lady the Gloria Patri is sung as below and then the office continues with the next reading, then skip the third Responsory (Suscipe) and finish with the Te Deum.}

\rubrics{Feasts of Our Lady in Advent include the Immaculate Conception (8th December) and Our Lady of Guadalupe (12th December). The Advent Office is also employed for the feast of the Annunciation (25th March).}

\gresetinitiallines0

\gregorioscore{gabc/re--ave_maria_gloria_patri--sandhofe}

\gresetinitiallines1

\bigskip

\Vbar. Jube Dómine, benedícere.

\emph{Benedictio.} Per Vírginem matrem concédat nobis Dóminus salútem et pacem. Amen.

\subsection{Lectio 3, Luc 1:34-38}

\startParallel
Dixit autem Maria ad Angelum: Quomodo fiet istud, quoniam virum non cognosco?
Et respondens Angelus dixit ei: Spiritus Sanctus superveniet in te, et virtus Altissimi obumbrabit tibi. Ideoque et quod nascetur ex te Sanctum, vocabitur Filius Dei.
Et ecce Elisabeth cognata tua, et ipsa concepit filium in senectute sua: et hic mensis sextus est illi, quae vocatur sterilis:
Quia non erit impossibile apud Deum omne verbum.
Dixit autem Maria: Ecce ancilla Domini: fiat mihi secundum verbum tuum.
\switchcolumn
\tr{And Mary said to the angel: How shall this be done, because I know not man? And the angel answering, said to her: The Holy Ghost shall come upon thee and the power of the Most High shall overshadow thee. And therefore also the Holy which shall be born of thee shall be called the Son of God. And behold thy cousin Elizabeth, she also hath conceived a son in her old age: and this is the sixth month with her that is called barren. Because no word shall be impossible with God. And Mary said: Behold the handmaid of the Lord: be it done to me according to thy word. And the angel departed from her.}
\switchcolumn
\Vbar. Tu autem, Dómine, miserére nobis.

\Rbar. Deo grátias.
\switchcolumn
\tr{Do thou, O Lord, have mercy on us.

Thanks be to God.}
\stopParallel

\subsection{Third Responsory}

\gregorioscore{gabc/re--suscipe_verbum--sandhofe}

\rubrics{Thus ends Matins. Go straight to Lauds. If you're not ready for Lauds, you can finish up the Unofficial Conclusion, page \pageref{unoff-conc}.}

\chapter{Te Deum}

\label{tedeum}

\rubrics{This is sung most of the year. During Lent and Advent this is not said and the Third Responsory is said in its place, except on feasts of Our Lady (the Feast of the Immaculate Conception on the 8th December, Our Lady of Guadalupe 12 December, the Feast of Candlemas on 2 February, Our Lady of Lourdes on 11 February and the Annunciation 25 March.) when the Te Deum is sung.}

\subsection{Simple Tone}

\gregorioscore{gabc/hy--te_deum_(simple_tone)--solesmes}

\rubrics{Thus ends Matins. Go straight to Lauds. If you're not ready for Lauds, you can finish up the Unofficial Conclusion, page \pageref{unoff-conc}.}

\subsection{Solemn Tone}

\gregorioscore{gabc/hy--te_deum_(solemn_tone)--solesmes}

\rubrics{Thus ends Matins. Go straight to Lauds. If you're not ready for Lauds, you can finish up the Unofficial Conclusion, page \pageref{unoff-conc}.}

\chapter{Unofficial Conclusion}

\label{unoff-conc}
\rubrics{It is usual for Lauds to follow immediately after the third responsory or the Te Deum. However, if Matins and Lauds are not said together the hour may be concluded thus in private recitation.}

\rubrics{The Baronius Press book only gives the Oration from Outside Christmas, but mentions that it is taken from Lauds, so I have included the Christmas variation from Lauds here too. Since it is an unofficial way to finish Matins, I guess it's okay to guess.}


\gresetinitiallines0
\gregorioscore{gabc/domine-exaudi}
\gresetinitiallines1


\subsection{Outside Christmas:}

\startParallel
Deus, qui de beátae Maríae Vírginis útero Verbum tuum, Angelo nuntiánte, carnem suscípere voluísti: praesta supplícibus tuis; ut, qui vere eam Genitrícem Dei crédimus, ejus apud te intercessiónibus adjuvémur.
Per eúndem Dóminum nostrum Jesum Christum Fílium tuum, qui tecum vivit et regnat in unitáte Spíritus Sancti, Deus, per ómnia sǽcula sæculórum.

\Rbar. Amen.
\switchcolumn
\tr{O God, Thou hast willed that at the message of an angel Thy Word should take flesh in the womb of the Blessed Virgin Mary; grant to Thy suppliant people, that we, who believe her to be truly the Mother of God, may be helped by her intercession with Thee. Through Christ our Lord. Amen.}
\stopParallel

\subsection{During Christmas:}

\startParallel
Deus, qui salútis ætérnæ, beátæ Maríæ virginitáte fecúnda, humáno géneri prǽmia præstitísti: tríbue, quǽsumus; ut ipsam pro nobis intercédere sentiámus, per quam merúimus auctórem vitæ suscípere, Dóminum nostrum Jesum Christum Fílium tuum:
Qui tecum vivit et regnat in unitáte Spíritus Sancti, Deus, per ómnia sǽcula sæculórum.

\Rbar. Amen.
\switchcolumn
\tr{O God, by the fruitful virginity of Blessed Mary, Thou hast bestowed upon the human race the rewards of eternal salvation; grant, we beseech Thee, that we may feel the power of her intercession, through whom we have been made worthy to receive the Author of life, Thy Son. Who with Thee lives and reigns forever. Amen.}
\stopParallel





\section{Conclusion}

\startParallel
\Vbar. Dómine, exáudi oratiónem meam.

\Rbar. Et clamor meus ad te véniat.

\Vbar. Benedicámus Dómino.

\Rbar. Deo grátias.

\Vbar. Fidélium ánimæ per misericórdiam Dei requiéscant in pace.

\Rbar. Amen.
\switchcolumn
\tr{V. O Lord, hear my prayer.

R. And let my cry come unto thee.

V. Let us bless the Lord.

R. Thanks be to God.

V. May the souls of the faithful, through the mercy of God, rest in peace.

R. Amen.}
\stopParallel


