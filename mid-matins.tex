
\chapter{Oración Inicial}

\gresetinitiallines1

\gregorioscore{gabc/an--just-domine-labia-mea-esp}

\gregorioscore{deusinadj}

\chapter{Invitatorio}


%\gregorioscore{gabc/invitatory-avemaria}

%\gregorioscore{gabc/va--venite_exsultemus--hartker}

\section{Tono Simple}

\rubrics{El cantor entona esta antífona:}

\gregorioscore{gabc/invitatory-avemaria-init}

\tr{Dios te salve, María, llena eres de gracia, el Señor es contigo.}

\rubrics{El coro repite esta antífona.}

\rubrics{El cantor canta los versículos y el coro canta las antífonas entre los versículos, alternándose el canto de la antífona completa con el canto de la segunda parte solamente, \emph{Dóminus tecum}.}

\subsection{Salmo 94.}

\gregorioscore{gabc/94-7a-1}

\tr{Venid, regocijémonos en el Señor; cantemos con júbilo las alabanzas de Dios, Salvador nuestro. Corramos a presentarnos ante su acatamiento, dándole gracias, y entonándole himnos con júbilo.}

\gresetinitiallines0
\gregorioscore{gabc/invitatory-avemaria}

\tr{Dios te salve, María, llena eres de gracia, el Señor es contigo.}


\gregorioscore{gabc/94-7a-2}

\tr{Porque el Señor es el Dios grande, y un rey más grande que todos los dioses. Porque en su mano tiene toda la extensión de la tierra, y suyos son los más encumbrados montes.}

\gregorioscore{gabc/invitatory-dominustecum}

\gregorioscore{gabc/94-7a-3}

\tr{Suyo es el mar, y obra es de sus manos: y hechura de sus manos es la tierra. (de rodillas) Venid, pues, adorémosle, postrémonos: derramando lágrimas en la presencia del Señor que nos ha creado: Pues Él es el Señor Dios nuestro: y nosotros el pueblo a quien Él apacienta, y ovejas de su grey.}

\gregorioscore{gabc/invitatory-avemaria}

\gregorioscore{gabc/94-7a-4}

\tr{Hoy mismo, si oyereis su voz, guardaos de endurecer vuestros corazones, como sucedió, dice el Señor, cuando provocaron mi ira, poniéndome a prueba en el desierto, en donde vuestros padres me tentaron, me probaron, y vieron mis obras.}

\gregorioscore{gabc/invitatory-dominustecum}

\gregorioscore{gabc/94-7a-5}

\tr{Por espacio de cuarenta años estuve irritado contra esta generación, y dije: siempre está descarriado el corazón de este pueblo. Ellos no conocieron mis caminos; por lo que juré airado que no entrarían en mi reposo.}

\gregorioscore{gabc/invitatory-avemaria}

\gregorioscore{gabc/94-7a-6}

\tr{Gloria al Padre, al Hijo, y al Espíritu Santo. Como era en el principio, ahora y siempre, por los siglos de los siglos. Amén.}

\gregorioscore{gabc/invitatory-dominusavedominus}

\gresetinitiallines1

\chapter{Himno}

\subsection{Melodía del Oficio Parvo}

\gregorioscore{gabc/hy--quem_terra_sidera}

\setcounter{versecount}{0}
\startParallel
\latin{Quem terra, pontus, sídera
	Colunt, adórant, prædicant,
	Trinam regéntem máchinam,
	Claustrum Maríæ báiulat.}
\vern{Aquel a quien la tierra, el mar y las estrellas
	veneran, adoran y anuncian;
	el que gobierna cielos, tierra y abismos,
	reside en el seno de María.}

\latin{Cui luna, sol, et ómnia
	Desérviunt per témpora,
	Perfúsa cæli grátia,
	Gestant puéllæ víscera.}
\vern{Al que el sol, la luna y todos los elementos
	sirven en el tiempo, le llevan
	las entrañas de una virgen
	llena de gracia celestial.}

\latin{Beáta Mater múnere,
	Cuius supérnus ártifex
	Mundum pugíllo cóntinens,
	Ventris sub arca clausus est.}
\vern{¡Oh Madre dichosa! En el arca de su seno,
	por un prodigio de la gracia,
	se encierra el supremo Artífice
	que en sus manos sostiene el orbe.}

\latin{Beáta cæli núntio,
	Fœcúnda sancto Spíritu,
	Desiderátus géntibus,
	Cuius per alvum fusus est.}
\vern{Dichosa aquella que al anuncio
	del mensajero celestial fue fecundada
	por el Espíritu Santo; por cuyo seno se nos dio
	el deseado de todos los pueblos.}

\latin{Iesu, tibi sit glória,
Qui natus es de Vírgine,
Cum Patre, et almo Spíritu
In sempitérna sǽcula.
\vern{Gloria a ti, oh Jesús, nacido de la Virgen,
	juntamente con el Padre
	y el Espíritu Santo,
	por los siglos de los siglos.
}

\latin{Amen.}
\vern{Amén.}
\stopParallel


\chapter{Nocturnos}

\section{Primer Nocturno (domingo, lunes y jueves)}

\gregorioscore{gabc/benedictatu-init}

\subsection{Salmo 8}

\gresetinitiallines0
\gregorioscore{gabc/8-4A}

\setcounter{versecount}{1}
\startParallel
\input psalms/8-4A
\stopParallel

\gregorioscore{gabc/benedictatu}

\bigskip


\gresetinitiallines1
\gregorioscore{gabc/an--sicut_myrrha--nocturnale_2002-init}

\subsection{Salmo 18}

\gresetinitiallines0
\gregorioscore{psalms/18-4A}

\setcounter{versecount}{1}
\startParallel
\input psalms/18-4Avv
\stopParallel

\gregorioscore{gabc/an--sicut_myrrha--nocturnale_2002}

\bigskip


\gresetinitiallines1
\gregorioscore{gabc/an--ante_torum--nocturnale_2002-init}

\subsection{Salmo 23}

\gresetinitiallines0
\gregorioscore{psalms/23-4A}

\setcounter{versecount}{1}
\startParallel
\input psalms/23-4Avv
\stopParallel

\gregorioscore{gabc/an--ante_torum--nocturnale_2002}

\rubrics{Con esto finaliza la Salmodia --- debes ir ahora al Versículo \emph{Diffusa est gratia} page \pageref{diffusa}.}

\section{Segundo Nocturno (martes y viernes)}

\gresetinitiallines1
\gregorioscore{gabc/an--specie_tua_(ant)--nocturnale_2002-init}

\subsection{Salmo 44}

\gresetinitiallines0
\gregorioscore{gabc/44-7c}

\setcounter{versecount}{1}
\startParallel
\input psalms/44-7c
\stopParallel

\setcounter{versecount}{0}

\gregorioscore{gabc/an--specie_tua_(ant)--nocturnale_2002}

\bigskip


\gresetinitiallines1
\gregorioscore{gabc/an--adjuvabit_eam--nocturnale_2002-init}


\subsection{Salmo 45}

\gresetinitiallines0
\gregorioscore{gabc/45-7c}

\startParallel
\setcounter{versecount}{1}
\input psalms/45-7c
\stopParallel

\gregorioscore{gabc/an--adjuvabit_eam--nocturnale_2002}

\bigskip


\gresetinitiallines1
\gregorioscore{gabc/an--sicut_laetantium--nocturnale_2002-init}

\subsection{Salmo 86}

\gresetinitiallines0
\gregorioscore{gabc/86-7c}

\startParallel
\setcounter{versecount}{1}
\input psalms/86-7c
\stopParallel



\gregorioscore{gabc/an--sicut_laetantium--nocturnale_2002}

\rubrics{Con esto finaliza la Salmodia --- debes ir ahora al Versículo \emph{Diffusa est gratia} page \pageref{diffusa}.}

\newpage

\section{Tercer Nocturno (miércoles y sábado)}

\gresetinitiallines1
\gregorioscore{gabc/an--gaude_maria_virgo--nocturnale_2002-init}

\subsection{Salmo 95}

\gresetinitiallines0
\gregorioscore{gabc/95-4c}

\startParallel
\setcounter{versecount}{1}
\input psalms/95-4c
\stopParallel

\gregorioscore{gabc/an--gaude_maria_virgo--nocturnale_2002}

\bigskip

\gresetinitiallines1
\gregorioscore{gabc/an--dignare_me_laudare--nocturnale_2002}

\subsection{Salmo 96}

\gresetinitiallines0
\gregorioscore{gabc/96-4c}

\startParallel
\setcounter{versecount}{1}
\input psalms/96-4c
\stopParallel

\gregorioscore{gabc/an--dignare_me_laudare--nocturnale_2002}

\bigskip

\rubrics{A continuación sigue el Salmo 97.
	La mayor parte del tiempo la antífona es \emph{Post partum}, sin embargo, en el Tiempo de Adviento y en la fiesta de la Anunciación el 25 de marzo, la antífona es \emph{Angelus Domini}, página \pageref{noc3ps3adv}.}

\gresetinitiallines1
\gregorioscore{gabc/an--post_partum-init}

\subsection{Salmo 97}

\gresetinitiallines0
\gregorioscore{gabc/97-4Astar}

\startParallel
\setcounter{versecount}{1}
\input psalms/97-4Astar
\stopParallel

\gregorioscore{gabc/an--post_partum}

\bigskip

\label{noc3ps3adv}
\rubrics{Así es como este tercer Salmo del tercer nocturno se canta en Adviento y en la fiesta de la Anunciación:}

\gresetinitiallines1
%\gregorioscore{gabc/an--angelus_domini_(ad_matutinum_in_adv._et_ann.tionis)--nocturnale_2002}

\gregorioscore{gabc/an--angelus_domini_nuntiavit--solesmes}

\subsection{Salmo 97}

\gresetinitiallines0
\gregorioscore{gabc/97-1f}

\startParallel
\setcounter{versecount}{2}
\input psalms/97-1f
\switchcolumn
\tr{Cantad al Señor un cántico nuevo, porque ha hecho maravillas:

	Su diestra le ha dado la victoria, su santo brazo.

	El Señor da a conocer su victoria, revela a las naciones su justicia:

	Se acordó de su misericordia y su fidelidad en favor de la casa de Israel.

	Los confines de la tierra han contemplado la victoria de nuestro Dios.

	Aclamad al Señor, tierra entera; gritad, vitoread, tocad:

	Tañed la cítara para el Señor, suenen los instrumentos: con clarines y al son de trompetas,

	Aclamad al Rey y Señor. Retumbe el mar y cuanto contiene, la tierra y cuantos la habitan;

	Aplaudan los ríos, aclamen los montes al Señor, que llega para regir la tierra.

	Regirá el orbe con justicia * y los pueblos con rectitud.

	Gloria al Padre...}
\stopParallel

\gregorioscore{gabc/an--angelus_domini_nuntiavit--solesmes}

%\gregorioscore{gabc/an--angelus_domini_(ad_matutinum_in_adv._et_ann.tionis)--nocturnale_2002}

\setcounter{versecount}{0}

\chapter{Versículos y Padrenuestro}
\label{diffusa}

\rubrics{Las Maitines continúan con los siguientes versos.}

\gregorioscore{gabc/diffusa-est}

%\versicle Diffúsa est grátia in lábiis tuis.

%\response Proptérea benedíxit te Deus in aetérnum.

%Pater noster (secreto usque ad)

%\versicle Et ne nos indúcas in tentatiónem.

%\response Sed líbera nos a malo.

(Absolución)

\startParallel
\latin{Précibus et méritis beátæ Maríæ semper Vírginis et ómnium Sanctórum, perdúcat nos Dóminus ad regna cælórum. Amen.}
\vern{Por las preces y los méritos de la Bienaventurada siempre Virgen María y de todos los Santos, el Señor nos conduzca a la unión de su divino Hijo. Amén.}
\stopParallel

\chapter{Lecciones y Responsorios}

%The Responsories are the big things. Most days you just sing two Responsories and finish with the Te Deum. In that case you would sing the first \textbf{without} the \emph{Gloria Patri}, then sing the second \textbf{with} the \emph{Gloria Patri}. In penitential times, such as Septuagesima, Lent and Advent, the \emph{Te Deum} is not sung. In that case you would sing the first two Responsories \textbf{without} a \emph{Gloria Patri} then finish off the third Responsory \textbf{with} a \emph{Gloria Patri}.

\section{Fuera del Tiempo de Adviento}

\rubrics{Si está presente un sacerdote, la primera línea será: \emph{Jube domne, benedícere.} Luego el sacerdote imparte la \emph{Benedictio}. En caso contrario, quien lidera reza la \emph{Benedictio} en Nombre del Señor.}

\versicle  Jube Dómine, benedícere.

\emph{Benedictio.} Nos cum prole pia benedícat Virgo María. Amen.

\subsection{Lección 1, Sir 24:11-13}

\startParallel
In ómnibus réquiem quaesívi, et in hereditáte Dómini morábor.
Tunc praecépit, et dixit mihi Creátor ómnium: et qui creávit me, requiévit in tabernáculo meo.
Et dixit mihi: In Jacob inhábita, et in Israël hereditáre, et in eléctis meis mitte radíces.
\switchcolumn
\tr{En todo busqué en dónde posar, y en la heredad del Señor fijé mi morada. Entonces el Criador de todas las cosas dio sus órdenes, y me habló, y el que a mí me dio el ser, estableció en mí su tabernáculo, y me dijo: Habita en Jacob, y sea Israel tu herencia, y arráigate en medio de mis escogidos.}
\switchcolumn
\versicle Tu autem, Dómine, miserére nobis.

\response Deo grátias.
\switchcolumn
\tr{\versicle Tú, Señor, ten piedad de nosotros.

	\response Demos gracias a Dios.}
\stopParallel

\subsection{Primer Responsorio}

\gresetinitiallines1
\gregorioscore{gabc/re--sancta_et_immaculata--sandhofe}

\versicle Jube Dómine, benedícere.

\emph{Benedictio.} Ipsa Virgo Vírginum intercédat pro nobis ad Dóminum. Amen.

\subsection{Lección 2, Sir 24:15-16}

\startParallel
Et sic in Sion firmáta sum, et in civitáte sanctificáta simíliter requiévi, et in Jerúsalem potéstas mea.
Et radicávi in pópulo honorificáto, et in parte Dei mei heréditas illíus, et in plenitúdine sanctórum deténtio mea.
\switchcolumn
\tr{Y así fijé mi estancia en el monte de Sion, y fue el lugar de mi reposo la ciudad santa, y en Jerusalén está mi trono.
	Y me arraigué en un pueblo glorioso, y en la porción de mi Dios, la cual es su herencia, y mi habitación en la plena reunión de los santos.}
\switchcolumn
\versicle Tu autem, Dómine, miserére nobis.

\response Deo grátias.
\switchcolumn
\tr{\versicle Tú, Señor, ten piedad de nosotros.

	\response Demos gracias a Dios.}
\stopParallel

\subsection{Segundo Responsorio}

\gregorioscore{gabc/re--beata_es_virgo_maria--baronius}

\rubrics{Fuera de Septuagésima y Cuaresma, corresponde cantar el \emph{Gloria Patri}. Durante Septuagésima y Cuaresma, omitir el \emph{Gloria Patri} en este responsorio y continuar en \emph{Jube Domine/domne}.}

\gregorioscore{gabc/re--beata--gloriapatri}

\versicle Jube Dómine, benedícere.

\emph{Benedictio.} Per Vírginem matrem concédat nobis Dóminus salútem et pacem. Amen.

\subsection{Lección 3, Sir 24:17-20}

\startParallel
Quasi cedrus exaltáta sum in Líbano, et quasi cypréssus in monte Sion:
Quasi palma exaltáta sum in Cades, et quasi plantátio rosae in Jéricho:
Quasi olíva speciósa in campis, et quasi plátanus exaltáta sum juxta aquam in platéis.
Sicut cinnamómum et bálsamum aromatízans odórem dedi; quasi myrrha elécta dedi suavitátem odóris.
\switchcolumn
\tr{Elevada estoy cual cedro sobre el Líbano, y cual ciprés sobre el monte de Sion. Extendí mis ramas como una palma de Cades y como el rosal plantado en Jericó. Me alcé como un hermoso olivo en los campos, y como el plátano en las plazas junto al agua. Como el cinamomo y el bálsamo aromático despedí fragancia; como mirra escogida exhalé suave olor.}
\switchcolumn
\versicle Tu autem, Dómine, miserére nobis.

\response Deo grátias.
\switchcolumn
\tr{\versicle Tú, Señor, ten piedad de nosotros.

	\response Demos gracias a Dios.}
\stopParallel

\rubrics{Fuera de Septuagésima y Cuaresma, así como también en las Fiestas de la Santísima Virgen María, continuar con el \emph{Te Deum}, página \pageref{tedeum}. Durante Septuagésima y Cuaresma, concluir con el Tercer Responsorio \emph{Felix namque} en reemplazo del Te Deum.}

\subsection{Tercer Responsorio}

\gregorioscore{gabc/re--felix_namque--hartker}

\rubrics{De esta forma terminan las Maitines. Usualmente, son seguidas por las Laudes. Pero en esta versión hemos recogido una conclusión no oficial de las Maitines que se transcribe a continuación.}

\section{Durante el Tiempo de Adviento}

\versicle Jube Dómine, benedícere.

\emph{Benedictio.} Nos cum prole pia benedícat Virgo María. Amen.

\subsection{Lección 1, Luc 1:26-28}

\startParallel
Missus est Angelus Gábriel a Deo in civitátem Galilaéae, cui nomen Názareth,
Ad Vírginem desponsátam viro, cui nomen erat Joseph, de domo David: et nomen Vírginis María.
Et ingréssus Angelus ad eam dixit: Ave grátia plena: Dóminus tecum: benedícta tu in muliéribus.
\switchcolumn
\tr{El Ángel Gabriel fue enviado por Dios a Nazaret, ciudad de Galilea, A una Virgen desposada con un varón de la casa de David, llamado José, y el nombre de la Virgen era María. Y entrando el Ángel adonde ella estaba, le dijo: Ave, oh llena de gracia; el Señor es contigo; bendita Tú eres entre las mujeres.}
\switchcolumn
\versicle Tu autem, Dómine, miserére nobis.

\response Deo grátias.
\switchcolumn
\tr{\versicle Tú, Señor, ten piedad de nosotros.

	\response Demos gracias a Dios.}
\stopParallel


\subsection{Primer Responsorio}

\gregorioscore{gabc/re--missus_est_gabriel--sandhofe}

\versicle Jube Dómine, benedícere.

\emph{Benedictio.} Ipsa Virgo Vírginum intercédat pro nobis ad Dóminum. Amen.

\subsection{Lección 2, Luc 1:29-33}

\startParallel
Quae cum audisset, turbata est in sermone ejus, et cogitabat qualis esset ista salutatio.
Et ait Angelus ei: Ne timeas, Maria: invenisti enim gratiam apud Deum:
Ecce concipies in utero, et paries filium, et vocabis nomen ejus Jesum:
Hic erit magnus, et Filius Altissimi vocabitur, et dabit illi Dominus Deus sedem David patris ejus: et regnabit in domo Iacob in aeternum,
Et regni ejus non erit finis.
\switchcolumn
\tr{Al oír tales palabras, la Virgen se turbó, y se puso a considerar qué significaría una tal salutación. Mas el Ángel le dijo: ¡Oh María! no temas; porque has hallado gracia en los ojos de Dios; He aquí que has de concebir en tu seno, y darás a luz un hijo, a quien pondrás por nombre Jesús. Este será grande, y será llamado Hijo del Altísimo, al cual el Señor Dios dará el trono de su padre David, y reinará en la casa de Jacob eternamente,y su reino no tendrá fin.}
\switchcolumn
\versicle Tu autem, Dómine, miserére nobis.

\response Deo grátias.
\switchcolumn
\tr{\versicle Tú, Señor, ten piedad de nosotros.

	\response Demos gracias a Dios.}
\stopParallel

\subsection{Segundo Responsorio}

\gregorioscore{gabc/re--ave_maria--sandhofe}

\bigskip

\rubrics{En las Fiestas de Nuestra Señora, el Gloria se canta como se indica abajo, luego se omite el Tercer Responsorio (Suscipe) y se canta en cambio el Te Deum.}

\rubrics{Las Fiestas de Nuestra Señora incluyen la Inmaculada Concepción (8 de diciembre) y Nuestra Señora de Guadalupe (12 de diciembre). Igualmente, esta sección de Adviento se debe emplear el día de la FIesta de la Anunciación (25 de marzo).}

\gresetinitiallines0

\gregorioscore{gabc/re--ave_maria_gloria_patri--sandhofe}

\gresetinitiallines1

\bigskip

\versicle Jube Dómine, benedícere.

\emph{Benedictio.} Per Vírginem matrem concédat nobis Dóminus salútem et pacem. Amen.

\subsection{Lección 3, Luc 1:34-38}

\startParallel
Dixit autem Maria ad Angelum: Quomodo fiet istud, quoniam virum non cognosco?
Et respondens Angelus dixit ei: Spiritus Sanctus superveniet in te, et virtus Altissimi obumbrabit tibi. Ideoque et quod nascetur ex te Sanctum, vocabitur Filius Dei.
Et ecce Elisabeth cognata tua, et ipsa concepit filium in senectute sua: et hic mensis sextus est illi, quae vocatur sterilis:
Quia non erit impossibile apud Deum omne verbum.
Dixit autem Maria: Ecce ancilla Domini: fiat mihi secundum verbum tuum.
\switchcolumn
\tr{Mas, María dijo al Ángel: ¿Cómo ha de ser eso? Pues Yo no conozco varón alguno. El Ángel en respuesta, le dijo: El Espíritu Santo descenderá sobre ti, y la virtud del Altísimo te cubrirá con su sombra; por lo cual el fruto santo que de ti nacerá se llamará Hijo de Dios. Y ahí tienes a tu parienta Isabel, que en su vejez ha concebido un hijo; y la que se llamaba estéril, hoy cuenta ya el sexto mes; ya que para Dios nada es imposible. Entonces dijo María: He aquí la esclava del Señor, hágase en mí según tu palabra.}
\switchcolumn
\versicle Tu autem, Dómine, miserére nobis.

\response Deo grátias.
\switchcolumn
\tr{\versicle Tú, Señor, ten piedad de nosotros.

	\response Demos gracias a Dios.}
\stopParallel

\subsection{Tercer Responsorio}

\gregorioscore{gabc/re--suscipe_verbum--sandhofe}

\rubrics{Fin de las Maitines. Es usual que a continuación se recen las Laudes (no incluidas en este libro). En todo caso, se incluye una conclusión no oficial de las Maitines que figura en algunas ediciones del Pequeño Oficio, página \pageref{unoff-conc}.}

\chapter{Te Deum}

\label{tedeum}

\rubrics{
	El Te Deum se canta la mayor parte del año. Durante Adviento y Cuaresma el Te Deum se reemplaza por el Tercer Responsorio, exceptuadas las fiestas de Nuestra Señora (Inmaculada Concepción, 8 de diciembre, Nuestra Señora de Guadalupe 12 de diciembre, Nuestra Señora de la Candelaria 2 de febrero, Nuestra Señora de Luordes, 11 de febrero y la Anunciación, 25 de marzo). En todas estas fechas se canta el Te Deum.}

\subsection{Simple Tone}

\gregorioscore{gabc/hy--te_deum_(simple_tone)--solesmes}

\rubrics{Fin de las Maitines. Es usual que a continuación se recen las Laudes (no incluidas en este libro). En todo caso, se incluye una conclusión no oficial de las Maitines que figura en algunas ediciones del Pequeño Oficio, página \pageref{unoff-conc}.}

\subsection{Tono Solemne}

\gregorioscore{gabc/hy--te_deum_(solemn_tone)--solesmes}

\rubrics{Fin de las Maitines. Es usual que a continuación se recen las Laudes (no incluidas en este libro). En todo caso, se incluye una conclusión no oficial de las Maitines que figura en algunas ediciones del Pequeño Oficio, página \pageref{unoff-conc}.}

\chapter{Conclusión No Oficial}

\label{unoff-conc}
\rubrics{Como se mencionó, es usual que las Maitines sean seguidas de las Laudes. Se ofrece a continuación una conclusión no oficial de las Maitines que, recitada privadamente puede ser empleada para concluir esta Hora del Pequeño Oficio.}


\gresetinitiallines0
\gregorioscore{gabc/domine-exaudi}
\gresetinitiallines1


\subsection{Outside Christmas:}

\startParallel
Deus, qui de beátae Maríae Vírginis útero Verbum tuum, Angelo nuntiánte, carnem suscípere voluísti: praesta supplícibus tuis; ut, qui vere eam Genitrícem Dei crédimus, ejus apud te intercessiónibus adjuvémur.
Per eúndem Dóminum nostrum Jesum Christum Fílium tuum, qui tecum vivit et regnat in unitáte Spíritus Sancti, Deus, per ómnia sǽcula sæculórum.

\response Amen.
\switchcolumn
\tr{Dios, que quisiste que a la palabra del Ángel se encarnase tu Verbo en el seno de la bienaventurada Virgen María: haz, te suplicamos, que cuantos creemos que es verdaderamente Madre de Dios, seamos ayudados delante de ti con su intercesión. Por el mismo Señor Nuestro Jesucristo, tu Hijo, que vive y reina en la unidad del Espíritu Santo, Dios, por todos los siglos de los siglos. Amén.}
\stopParallel

\subsection{During Christmas:}

\startParallel
Deus, qui salútis ætérnæ, beátæ Maríæ virginitáte fecúnda, humáno géneri prǽmia præstitísti: tríbue, quǽsumus; ut ipsam pro nobis intercédere sentiámus, per quam merúimus auctórem vitæ suscípere, Dóminum nostrum Jesum Christum Fílium tuum:
Qui tecum vivit et regnat in unitáte Spíritus Sancti, Deus, per ómnia sǽcula sæculórum.

\response Amen.
\switchcolumn
\tr{¡Oh Dios, que por la maternidad virginal de santa María entregaste al género humano los tesoros de la salvación eterna!, concédenos la intercesión de la Madre de nuestro Redentor. Que vive y reina en unión del Espíritu Santo, Dios, por todos los siglos de los siglos. Amén.}
\stopParallel





\section{Conclusión}

\startParallel
\versicle Dómine, exáudi oratiónem meam.

\response Et clamor meus ad te véniat.

\versicle Benedicámus Dómino.

\response Deo grátias.

\versicle Fidélium ánimæ per misericórdiam Dei requiéscant in pace.

\response Amen.
\switchcolumn
\tr{\versicle Señor, escucha mi oración.

	\response Y llegue a ti mi clamor.

	\versicle Bendigamos al Señor.

	\response Demos gracias a Dios.

	\versicle Que las almas de los fieles, por la misericordia de Dios, descansen en paz.

	\response Amen.}
\stopParallel


