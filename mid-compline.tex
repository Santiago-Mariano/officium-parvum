
\chapter{Oración Inicial}


\rubrics{Hacer la señal de la Cruz con el pulgar sobre el corazón:}

\gresetinitiallines0
\gregorioscore{converte}

\gregorioscore{etaverte}

%\noindent \Vbar.\ 
%\Rbar.\ 

\rubrics{Signándose:}

\gresetinitiallines1
\gregorioscore{deusinadjengl}

\tr{\Vbar.\ Oh Dios, ven en mi auxilio.
\Rbar.\ Señor, date prisa en socorrerme.
Gloria al Padre\ldots}

\chapter{Salmodia}

\rubrics{Se sugiere sentarse tras el primer asterisco en cada salmo y ponerse de pie en el último asterisco y así estar preparado para hacer una reverencia en el \emph{Gloria Patri}, levantarse nuevamente en el \emph{Sicut erat}, y luego permanecer de pie hasta el primer asterisco del siguiente salmo.}

\section{Salmo 128}

\gresetinitiallines0
\gregorioscore{psalms/128v1-roman}

\setcounter{versecount}{2}
\input psalms/128vv-roman

\section*{Salmo 129}
\gregorioscore{psalms/129v1-roman}

\setcounter{versecount}{2}
\input psalms/129vv-roman

\section*{Salmo 130}
\gregorioscore{psalms/130v1-roman}

\setcounter{versecount}{2}
\input psalms/130vv-roman

\setcounter{enumi}{0}

\setcounter{versecount}{0}


\chapter{Himno}

\label{memento}
\gresetinitiallines1
%\gresetfirstlineaboveinitial{2}{2}
\gregorioscore{mementolit}

%\begin{multicols}{2}

%\noindent 1.\ Acuérdate, oh Creador de todas las cosas,\\
%que un día tomaste, al nacer\\
%del seno de una Virgen sagrada,\\
%la forma de nuestro cuerpo.
%
%\smallskip 
%
%\noindent 2.\ Oh María, Madre de la gracia,\\
%dulce Madre de misericordia,\\
%protégenos del enemigo maligno,\\
%y recíbenos en la hora de la muerte.
%
%\smallskip
%
%\noindent 3.\ Gloria a ti, oh Jesús, nacido de la Virgen,\\
%juntamente con el Padre\\
%y el Espíritu Santo,\\
%por los siglos de los siglos. Amén.
%
%%\end{multicols}

\subsection{Pequeño Capítulo}

\rubrics{Fuera del Tiempo de Adviento:}

\startParallel
\latin{Ego mater pulchrae dilectiónis, et timóris, et agnitiónis, et sanctae spei.}
\vern{Yo soy la Madre del amor hermoso, y del temor, y del conocimiento y de la santa esperanza.}
\latin{\Rbar. Deo gratias.}
\vern{Demos gracias a Dios.}
\latin{\Vbar. Ora pro nobis, sancta Dei Génitrix}
\vern{Ruega por nosotros Santa Madre de Dios.}
\latin{\Rbar. Ut digni efficiámur promissiónibus Christi}
\vern{Para que seamos dignos de alcanzar las promesas de Cristo}
\stopParallel

\rubrics{Durante el Tiempo de Adviento:}

\startParallel
\latin{Ecce Virgo concípiet, et páriet fílium, et vocábitur nomen ejus Emmánuel. Butýrum et mel cómedet, ut sciat reprobáre malum, et elígere bonum.}
\vern{Mirad: la Virgen está encinta y da a luz un hijo, y lo pone por nombre «Dios-con-nosotros». Comerá requesón con miel, hasta que aprenda a rechazar el mal y escoger el bien.}
\latin{\Rbar. Deo grátias.}
\vern{Demos gracias a Dios.}
\latin{\Vbar. Angelus Dómini nuntiávit Maríae.}
\vern{El ángel del Señor anunció a María.}
\latin{\Rbar. Et concépit de Spíritu Sancto.}
\vern{Y concibió del Espíritu Santo.}
\stopParallel

\chapter{Cántico}

\section*{Fuera de los Tiempos de Adviento, Navidad y Pascua:}

\gregorioscore{gabc/an--sub_tuum_praesidium--vatican}

%\gregorioscore{subtuum}



\tr{Bajo tu amparo nos acogemos, santa Madre de Dios: no desprecies las súplicas que te dirigimos en nuestras necesidades; antes bien líbranos siempre de todo peligro, Virgen gloriosa y bendita.}


\gregorioscore{nuncdimittis7a}

\tr{2:29 Ahora, Señor, según tu promesa, puedes dejar a tu siervo irse en paz, porque mis ojos han visto a tu Salvador, A quien has presentado ante todos los pueblos: luz para alumbrar a las naciones, y gloria de tu pueblo de Israel. V. Gloria al Padre, al Hijo y al Espíritu Santo. R. Como era en el principio, ahora y siempre, por los siglos de los siglos. Amen.}


\rubrics{Repetir la antífona \emph{Sub tuum} todos juntos.}

\rubrics{Luego ir a la Oración Colecta en la página~\pageref{collect}.}

\subsection*{Durante el Tiempo de Adviento:}

%\gresetfirstlineaboveinitial{8}{8}
\gregorioscore{gabc/an--spiritus_sanctus}

\tr{El Espíritu Santo vendrá sobre ti, María. No temas, concebirás en tu vientre al Hijo de Dios. Aleluya.}

\gregorioscore{nuncdimittis8G}

\rubrics{Repetir la antífona \emph{Spiritus sanctus} todos juntos.}

\rubrics{Luego ir a la Oración Colecta en la página~\pageref{collect}.}


\subsection*{Durante el Tiempo de Navidad:}

%\gresetfirstlineaboveinitial{2}{2}
\gregorioscore{magnum}

\tr{¡Qué misteriosa filiación! * El seno de la Virgen se ha convertido en templo de Dios. No se ha manchado al tomar carne de ella. Todos los pueblos vendrán cantando: Gloria a ti, Señor.}

\gregorioscore{nuncdimittis2A}

\rubrics{Repetir la antífona \emph{Magnum haereditatis mysterium} todos juntos.}

\rubrics{Luego ir a la Oración Colecta en la página~\pageref{collect}.}



\subsection*{Durante el Tiempo de Pascua:}

%\gresetfirstlineaboveinitial{1}{1}
\gregorioscore{reginacaeliAlt}

\tr{Reina del cielo, alégrate, aleluya, porque el Señor, a quien has merecido llevar, aleluya, ha resucitado, según su palabra, aleluya. Ruega al Señor por nosotros, aleluya.}

\gregorioscore{nuncdimittis1D2}

\rubrics{Repetir la antífona \emph{Regina caeli} todos juntos.}


\section*{Oración Colecta}
\label{collect}

\startParallel
\latin{\Vbar.\ Dómine exáudi oratiónem meam.}
\vern{Señor, escucha mi oración.}
\latin{\Rbar.\ Et clamor meus ad te véniat.}
\vern{Y llegue a ti mi clamor}
\latin{Orémus.}
\vern{Oremos.}
\stopParallel

\bigskip

\rubrics{Fuera del Tiempo de Adviento:}

\startParallel
\latin{Beátae et gloriósae semper Vírginis Maríae, \dag\ 
	quaésumus Dómine, intercéssio gloriósa nos prótegat, *
	et ad vitam perdúcat aetérnam.
Per Dóminum nostrum Jesum Christum, Filium tuum: qui tecum vivit et regnat in unitáte Spíritus Sancti Deus, per ómnia sǽcula sæculórum. \Rbar . Amen.}
\vern{Te suplicamos, Señor, que nos proteja la gloriosa intercesión de la bienaventurada, y gloriosa siempre Virgen María, y que nos lleve a la vida eterna.
Por nuestro Señor Jesucristo, tu Hijo, que vive y reina contigo en la unidad del Espíritu Santo, Dios, por todos los siglos de los siglos. Amén.}
\stopParallel

\bigskip

\rubrics{Durante el Tiempo de Adviento:}

\startParallel
\latin{Deus, qui de beatae Mariae Virginis utero, Verbum tuum, Angelo nuntiante, carnem suscipere voluisti: \dag\ praesta supplicibus tuis; ut, qui vere eam Genetricem Dei credimus, * ejus apud te intercessionibus adjuvemur. Per  eumdem Dominum nostrum Jesum Christum, Filium tuum, qui tecum vivit et regnat in unitate Spiritus Sancti Deus, per omnia saecula saeculorum.}
\vern{Dios, que quisiste que a la palabra del Ángel se encarnase tu Verbo en el seno de la bienaventurada Virgen María: haz, te suplicamos, que cuantos creemos que es verdaderamente Madre de Dios, seamos ayudados delante de ti con su intercesión. Por el mismo Señor Nuestro Jesucristo, tu Hijo, que vive y reina en la unidad del Espíritu Santo, Dios, por todos los siglos de los siglos.}
\latin{\Rbar.\ Amen.}
\vern{Amén.}
\stopParallel

\bigskip

\rubrics{Finalizar con:}

\startParallel
\latin{\Vbar.\ Dómine exáudi oratiónem meam.}
\vern{Señor, escucha mi oración.}
\latin{\Rbar.\ Et clamor meus ad te véniat.}
\vern{Y llegue a ti mi clamor.}
\stopParallel

\ 

\gresetinitiallines0
\gregorioscore{gabc/BenedicamusDomino}

%\Vbar. Benedicámus Dómino.

%\Rbar. Deo grátias.

\ 

\startParallel
\latin{Benedicat et custodiat nos omnipotens et misericors Dominus, Pater, \grecross\ et Filius et Spiritus Sanctus.}
\vern{Que nos bendiga y nos guarde el Señor omniponte y misericordioso, Padre, Hijo y Espíritu Santo.}
\latin{\Rbar.\ Amen.}
\vern{Amén.}
\stopParallel

