\documentclass[12pt,a5paper,openany]{memoir}
\input stdsetup


\nouppercaseheads
\makepagestyle{mystyle}
\setlength{\headwidth}{\dimexpr\textwidth+\marginparsep+\marginparwidth\relax}
\makerunningwidth{mystyle}{\headwidth}
\makeatletter
\makepsmarks{mystyle}{%
\createmark{chapter}{right}{nonumber}{\@chapapp\ }{. x\ }
\createmark{part}{left}{nonumber}{}{}
%\createmark{part}{right}{shownumber}{}{. \space}
%\createmark{section}{left}{nonumber}{}{}
}
\makeatother

\makeevenhead{mystyle}{\itshape\leftmark}{}{}
\makeoddhead{mystyle}{}{}{\itshape\rightmark}
\makeevenfoot{mystyle}{\thepage}{}{}
\makeoddfoot{mystyle}{}{}{\thepage}
\makepagestyle{plain}
\makerunningwidth{plain}{\headwidth}
\makeevenfoot{plain}{\thepage}{}{}
\makeoddfoot{plain}{}{}{\thepage}

\pagestyle{mystyle}

\renewcommand{\sectionmark}[1]{}
\renewcommand{\subsectionmark}[1]{}

\begin{document}

\mytitle{Pequeño Oficio de la Bienaventurada Vírgen María}{Resumido}

\tableofcontents*

\chapter{Oraciones antes del oficio}

\startParallel
Áperi Dómine, os meum ad benedicéndum nomen sanctum tuum: munda quoque cor meum ab ómnibus vanis, pervérsis et aliénis cogitatiónibus; intelléctum illúmina, afféctum inflámma, ut digne, atténte ac devóte hoc Offícium recitáre váleam, et exaudíri mérear ante conspéctum divínæ Majestátis túæ. Per Christum Dóminum nostrum.

\Rbar. Amen.

\switchcolumn

\selectlanguage{spanish}
\tr{Abre, Señor, mi boca para bendecir tu santo nombre; limpia también mi corazón de todo pensamiento vano, malo y distraído; ilumina mi entendimiento, inflama mi sentimiento, para que pueda recitar dignamente este oficio con atención y devoción y merecer ser escuchado en la presencia de tu majestad divina. Por Cristo nuestro Señor.}

\tr{\Rbar. Amen.}

\switchcolumn*

\selectlanguage{latin}
Dómine, in unióne illíus divínæ intentiónis, qua ipse in terris laudes Deo persolvísti, has tibi Horas (vel hanc tibi Horam) persólvo.
\switchcolumn
\selectlanguage{spanish}
\tr{
	Amén.
	Oh Señor, en unión con aquella divina intención, con la que tu mismo hiciste alabanza a Dios en la tierra, te ofrezco a ti estas horas (o esta hora).}
\stopParallel

\part{Maitines}

\input mid-matins

% \part{Laudes}

% \input mid-laudes

\rubrics{Es recomendable finalizar la recitación pública del Pequeño Oficio con la correspondiente Antífona Mariana, página \pageref{marian-ants}.}

% \end{document}


% \part{Prima}

% \deusinadjref

% \input mid-prima

% \rubrics{Es recomendable finalizar la recitación pública del Pequeño Oficio con la correspondiente Antífona Mariana, página \pageref{marian-ants}.}

% \part{Tertia}

% \deusinadjref

% \input mid-terce

% \rubrics{Es recomendable finalizar la recitación pública del Pequeño Oficio con la correspondiente Antífona Mariana, página \pageref{marian-ants}.}

% \part{Sexta}

% \deusinadjref

% \input mid-sext

% \rubrics{Es recomendable finalizar la recitación pública del Pequeño Oficio con la correspondiente Antífona Mariana, página \pageref{marian-ants}.}

% \part{Nona}

% \deusinadjref

% \input mid-none

% \rubrics{Es recomendable finalizar la recitación pública del Pequeño Oficio con la correspondiente Antífona Mariana, página \pageref{marian-ants}.}

\part{Vesperae}

Deus in adjutorium as at the beginning of Lauds(OJO, NO PUEDE HABER REFERENCIA A LAUDES PORQUE NO HABRÁ LAUDES EN NUESTRO LIBRO), page \pageref{deus-memento}

\startParallel
\latin{Deus in adjutórium meum intende.}
\vern{Oh Dios, ven en mi auxilio.}

\latin{Dómine, ad adjuvándum me festína.}
\vern{Señor, date prisa en socorrerme.}

\latin{Glória Patri, et Fílio, et Spirítui Sancto. Sicut erat in princípio, et nunc et semper, et in saécula saeculórum. Amen.}
\vern{Gloria al Padre,  al Hijo y al Espíritu Santo. Como era en el principio, ahora y siempre, por los siglos de los siglos. Amen.}
\stopParallel

\input mid-vespers

\rubrics{Es recomendable finalizar la recitación pública del Pequeño Oficio con la correspondiente Antífona Mariana, página \pageref{marian-ants}.}

\part{Completorium}

%\renewcommand{\latin}[1]{\selectlanguage{latin}\ifnum \value{versecount}>0 \stepcounter{versecount} \theversecount .~\fi#1\switchcolumn\selectlanguage{spanish}}
%\renewcommand{\vern}[1]{\tr{#1}\switchcolumn}


\input mid-compline

\rubrics{La Hora de las Completas siempre concluye con la Antífona Mariana correspondiente, página \pageref{marian-ants}.}

\chapter{Antífona Mariana }

\label{marian-ants}
\input marianantiphon


\chapter{Oraciones para después del oficio}

\selectlanguage{latin}
\startParallel
Sacrosánctæ et indivíduæ Trinitáti, crucifíxi Dómini nostri Jesu Christi humanitáti, beatíssimæ et gloriosíssimæ sempérque Vírginis Maríæ fœcúndæ integritáti, et ómnium Sanctórum universitáti sit sempitérna laus, honor, virtus et glória ab omni creatúra, nobísque remíssio ómnium peccatórum, per infiníta sǽcula sæculórum.

\Rbar. Amen.

\switchcolumn

\selectlanguage{spanish}
\tr{Señor Jesucristo, a la fecunda virginidad de la santísima y gloriosísima María, siempre Virgen,, a toda la asamblea de los santos, sea atribuida toda alabanza, honor, poder y gloria por toda criatura; y a nosotros se nos conceda la remisión de todos nuestros pecados, por los siglos de los siglos.


	\Rbar. Amen.}

\switchcolumn*


\Vbar. Beáta víscera Maríæ Vírginis, quæ portavérunt ætérni Pátris Fílium.

\Rbar. Et beáta úbera, quæ lactavérunt Christum Dóminum.

\switchcolumn

\selectlanguage{spanish}
\tr{\Vbar. Bendito el vientre de la Virgen María, que portó al Hijo del Padre Eterno.


	\Rbar. Y benditos los pechos que alimentaron a Cristo nuestro Señor.}


\switchcolumn*
\selectlanguage{latin}

Pater, Ave et Credo in secreto.

\switchcolumn

\selectlanguage{spanish}
\tr{Padre Nuestro, Ave María y Credo, en silencio}

\switchcolumn*

\selectlanguage{latin}
Pater noster, qui es in cælis, sanctificétur nomen tuum: advéniat regnum tuum: fiat volúntas tua, sicut in cælo et in terra. Panem nostrum quotidiánum da nobis hódie: et dimítte nobis débita nostra, sicut et nos dimíttimus debitóribus nostris: et ne nos indúcas in tentatiónem: sed líbera nos a malo. Amen.

\switchcolumn

\selectlanguage{spanish}
\tr{Padre nuestro, que estás en el Cielo, santificado sea tu Nombre; venga a nosotros tu reino; hágase Tu voluntad, en la tierra como en el Cielo. Danos hoy nuestro pan de cada día; perdona nuestras ofensas, como también nosotros perdonamos a los que nos ofenden; no nos dejes caer en la tentación, y líbranos del mal. Amen.}

\switchcolumn*

\selectlanguage{latin}
Ave María, grátia plena; Dóminus tecum: benedícta tu in muliéribus, et benedíctus fructus ventris tui Jesus. Sancta María, Mater Dei, ora pro nobis peccatóribus, nunc et in hora mortis nostræ. Amen.


\switchcolumn

\selectlanguage{spanish}
\tr{Dios te salve María llena eres de gracia el Señor es contigo; bendita tú eres entre todas las mujeres, y bendito es el fruto de tu vientre, Jesús. Santa María, Madre de Dios, ruega por nosotros, pecadores, ahora y en la hora de nuestra muerte. Amen.}

\switchcolumn*

\selectlanguage{latin}
Credo in Deum, Patrem omnipoténtem, Creatórem cæli et terræ. Et in Jesum Christum, Fílium ejus únicum, Dóminum nostrum: qui concéptus est de Spíritu Sancto, natus ex María Vírgine, passus sub Póntio Piláto, crucifíxus, mórtuus, et sepúltus: descéndit ad ínferos; tértia die resurréxit a mórtuis; ascéndit ad cælos; sedet ad déxteram Dei Patris omnipoténtis: inde ventúrus est judicáre vivos et mórtuos. Credo in Spíritum Sanctum, sanctam Ecclésiam cathólicam, Sanctórum communiónem, remissiónem peccatórum, carnis resurrectiónem, vitam ætérnam. Amen.
\switchcolumn

\selectlanguage{spanish}
\tr{Creo en Dios, Padre todopoderoso, Creador del cielo y de la tierra. Creo en Jesucristo, su único Hijo, nuestro Señor, que fue concebido por obra y gracia del Espíritu Santo, nació de Santa María Virgen, padeció bajo el poder de Poncio Pilato, fue crucificado, muerto y sepultado, descendió a los infiernos, al tercer día resucitó de entre los muertos, subió a los cielos y está sentado a la derecha de Dios, Padre todopoderoso. Desde allí ha de venir a juzgar a vivos y muertos. Creo en el Espíritu Santo, la santa Iglesia católica, la comunión de los santos, el perdón de los pecados, la resurrección de la carne y la vida eterna. Amén.}

\stopParallel


\end{document}

