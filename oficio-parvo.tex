\documentclass[12pt,a5paper,openright]{memoir}
\input stdsetup



\begin{document}

% Primera página, portada

\thispagestyle{empty}


\null\vfill
\begin{center}
    \large
    \textsc{Officium Parvum Beatae Mariae Virginis}

    \bigskip

    \huge
    \textsc{\gregorian El Pequeño Oficio de la Bienaventurada Virgen María}
    % \textsc{El Pequeño Oficio de la Bienaventurada Virgen María}

    \bigskip
    \bigskip
    \bigskip


    \normalsize Maitines, Vísperas y Completas


    \footnotesize


\end{center}
\null\vfill

% Segunda página, contra-portada
\newpage
\thispagestyle{empty}

\null\vfill
\begin{center}

    \today

    www.cantatedomino.cl

    \bigskip
    \bigskip
    \bigskip
    \bigskip
    \bigskip
    \bigskip

    \footnotesize

    \textbf{Basado en el generoso trabajo de Verónica Brandt. }

    \bigskip
    \bigskip
    \bigskip

    \begin{minipage}{0.6\textwidth}


        %This work is licensed under the Creative Commons Attribution 4.0 International License. To view a copy of this license, visit

        \includegraphics[height=0.8\baselineskip]{by.pdf} Este trabajo está licenciado bajo la ``Creative Commons Attribution 4.0 International License''. Para ver una copia de esta licencia, visitar:
        http://creativecommons.org/licenses/by/4.0/.

    \end{minipage}


\end{center}
\null\vfill




\cleardoublepage


%Tabla de contenidos

\renewcommand{\thepage}{\roman{page}}

\pagenumbering{roman}
\setcounter{page}{1}

\tableofcontents*



%Cuerpo

\cleardoublepage

\renewcommand{\thepage}{\arabic{page}}

\pagenumbering{arabic}
\setcounter{page}{1}



\cleardoublepage
\chapter*{Introducción}
\addcontentsline{toc}{chapter}{Introducción}


(TEXTO GENERADO POR CHATGPT)

El Pequeño Oficio de la Bienaventurada Virgen María es una devoción tradicional dentro de la Iglesia Católica que consiste en la recitación diaria de oraciones y lecturas bíblicas dedicadas a la Virgen María. Esta práctica, arraigada en la historia de la Iglesia, ofrece múltiples beneficios espirituales, emocionales y comunitarios para aquellos fieles que deciden incorporarla en su vida de oración. A continuación, se exploran algunas de las razones por las cuales conviene rezar el Pequeño Oficio de la Bienaventurada Virgen María.

En primer lugar, rezar el Pequeño Oficio permite a los fieles honrar y profundizar su relación con la Virgen María, quien es modelo de fe, obediencia y humildad. La devoción mariana nos conecta con la madre de Jesucristo, facilitando una mayor comprensión y amor hacia Ella, lo que, a su vez, nos acerca más a su Hijo. Al reflexionar sobre las virtudes y el papel de María en la historia de la salvación, los fieles pueden encontrar inspiración para seguir su ejemplo en su propia vida espiritual.

Además, el Pequeño Oficio promueve la disciplina espiritual y la oración constante. Al comprometerse a rezar las horas del Oficio a lo largo del día, los fieles desarrollan una rutina de oración que estructura su día en torno a momentos de reflexión y conexión con Dios. Esta práctica regular de oración fortalece la fe, ofrece consuelo en momentos de dificultad y brinda paz interior al recordar constantemente la presencia y protección de Dios en nuestras vidas.

Otra ventaja de esta devoción es su papel en la comunidad de creyentes. Aunque se puede rezar individualmente, el Pequeño Oficio también se practica en grupo, fomentando la unidad y el apoyo mutuo entre los fieles. Esta experiencia comunitaria de oración refuerza los lazos entre los miembros de la Iglesia, promoviendo un sentido de pertenencia y solidaridad en el camino espiritual compartido.

El Pequeño Oficio de la Bienaventurada Virgen María también es accesible y flexible, adaptándose a las necesidades y circunstancias de cada fiel. Aunque tiene una estructura definida, los individuos pueden ajustar la frecuencia y la forma en que lo rezan, permitiendo que personas de diversas etapas de vida y con diferentes horarios integren esta práctica en su rutina diaria.

Finalmente, rezar el Pequeño Oficio es una forma de participar en una tradición viva que ha enriquecido la vida espiritual de incontables generaciones de católicos a lo largo de los siglos. Al unirse a esta corriente de fe, los fieles se conectan con la Iglesia universal, tanto pasada como presente, y contribuyen a la transmisión de esta rica herencia espiritual a futuras generaciones.

En conclusión, el Pequeño Oficio de la Bienaventurada Virgen María ofrece profundos beneficios espirituales, promoviendo el crecimiento personal en la fe, la disciplina en la oración, el fortalecimiento de la comunidad cristiana, la flexibilidad en la práctica devocional y la conexión con una tradición histórica. Por estas razones, rezar el Pequeño Oficio es una práctica enriquecedora y recomendable para todos los fieles que deseen profundizar su devoción mariana y su vida espiritual.





\cleardoublepage
\chapter*{Oraciones antes del oficio}
\addcontentsline{toc}{chapter}{Oraciones antes del oficio}

\startParallel
Áperi Dómine, os meum \redgrecross ad benedicéndum nomen sanctum tuum: munda quoque cor meum \redgrecross ab ómnibus vanis, pervérsis et aliénis cogitatiónibus; intelléctum illúmina, afféctum inflámma, ut digne, atténte ac devóte hoc Offícium recitáre váleam, et exaudíri mérear ante conspéctum divínæ Majestátis túæ. Per Christum Dóminum nostrum.

\response Amen.

\switchcolumn

\selectlanguage{spanish}
\tr{Abre, Señor, mi boca \redgrecross para bendecir tu santo nombre; limpia también mi corazón \redgrecross de todo pensamiento vano, malo y distraído; ilumina mi entendimiento, inflama mi sentimiento, para que pueda recitar dignamente este oficio con atención y devoción y merecer ser escuchado en la presencia de tu majestad divina. Por Cristo nuestro Señor.}

\tr{\response Amen.}

\switchcolumn*

\selectlanguage{latin}
Dómine, in unióne illíus divínæ intentiónis, qua ipse in terris laudes Deo persolvísti, has tibi Horas (vel hanc tibi Horam) persólvo.
\switchcolumn
\selectlanguage{spanish}
\tr{
    Amén.
    Oh Señor, en unión con aquella divina intención, con la que tu mismo hiciste alabanza a Dios en la tierra, te ofrezco a ti estas horas (o esta hora).}
\stopParallel




\cleardoublepage
\chapter*{Oraciones para después del oficio}
\addcontentsline{toc}{chapter}{Oraciones para después del oficio}

\selectlanguage{latin}
\startParallel
Sacrosánctæ et indivíduæ Trinitáti, crucifíxi Dómini nostri Jesu Christi humanitáti, beatíssimæ et gloriosíssimæ sempérque Vírginis Maríæ fœcúndæ integritáti, et ómnium Sanctórum universitáti sit sempitérna laus, honor, virtus et glória ab omni creatúra, nobísque remíssio ómnium peccatórum, per infiníta sǽcula sæculórum.

\response Amen.

\switchcolumn

\selectlanguage{spanish}
\tr{Señor Jesucristo, a la fecunda virginidad de la santísima y gloriosísima María, siempre Virgen,, a toda la asamblea de los santos, sea atribuida toda alabanza, honor, poder y gloria por toda criatura; y a nosotros se nos conceda la remisión de todos nuestros pecados, por los siglos de los siglos.


    \response Amen.}

\switchcolumn*


\versicle Beáta víscera Maríæ Vírginis, quæ portavérunt ætérni Pátris Fílium.

\response Et beáta úbera, quæ lactavérunt Christum Dóminum.

\switchcolumn

\selectlanguage{spanish}
\tr{\versicle Bendito el vientre de la Virgen María, que portó al Hijo del Padre Eterno.


    \response Y benditos los pechos que alimentaron a Cristo nuestro Señor.}


\switchcolumn*
\selectlanguage{latin}

Pater, Ave et Credo in secreto.

\switchcolumn

\selectlanguage{spanish}
\tr{Padre Nuestro, Ave María y Credo, en silencio}

\switchcolumn*

\selectlanguage{latin}
Pater noster, qui es in cælis, sanctificétur nomen tuum: advéniat regnum tuum: fiat volúntas tua, sicut in cælo et in terra. Panem nostrum quotidiánum da nobis hódie: et dimítte nobis débita nostra, sicut et nos dimíttimus debitóribus nostris: et ne nos indúcas in tentatiónem: sed líbera nos a malo. Amen.

\switchcolumn

\selectlanguage{spanish}
\tr{Padre nuestro, que estás en el Cielo, santificado sea tu Nombre; venga a nosotros tu reino; hágase Tu voluntad, en la tierra como en el Cielo. Danos hoy nuestro pan de cada día; perdona nuestras ofensas, como también nosotros perdonamos a los que nos ofenden; no nos dejes caer en la tentación, y líbranos del mal. Amen.}

\switchcolumn*

\selectlanguage{latin}
Ave María, grátia plena; Dóminus tecum: benedícta tu in muliéribus, et benedíctus fructus ventris tui Jesus. Sancta María, Mater Dei, ora pro nobis peccatóribus, nunc et in hora mortis nostræ. Amen.


\switchcolumn

\selectlanguage{spanish}
\tr{Dios te salve María llena eres de gracia el Señor es contigo; bendita tú eres entre todas las mujeres, y bendito es el fruto de tu vientre, Jesús. Santa María, Madre de Dios, ruega por nosotros, pecadores, ahora y en la hora de nuestra muerte. Amen.}

\switchcolumn*

\selectlanguage{latin}
Credo in Deum, Patrem omnipoténtem, Creatórem cæli et terræ. Et in Jesum Christum, Fílium ejus únicum, Dóminum nostrum: qui concéptus est de Spíritu Sancto, natus ex María Vírgine, passus sub Póntio Piláto, crucifíxus, mórtuus, et sepúltus: descéndit ad ínferos; tértia die resurréxit a mórtuis; ascéndit ad cælos; sedet ad déxteram Dei Patris omnipoténtis: inde ventúrus est judicáre vivos et mórtuos. Credo in Spíritum Sanctum, sanctam Ecclésiam cathólicam, Sanctórum communiónem, remissiónem peccatórum, carnis resurrectiónem, vitam ætérnam. Amen.
\switchcolumn

\selectlanguage{spanish}
\tr{Creo en Dios, Padre todopoderoso, Creador del cielo y de la tierra. Creo en Jesucristo, su único Hijo, nuestro Señor, que fue concebido por obra y gracia del Espíritu Santo, nació de Santa María Virgen, padeció bajo el poder de Poncio Pilato, fue crucificado, muerto y sepultado, descendió a los infiernos, al tercer día resucitó de entre los muertos, subió a los cielos y está sentado a la derecha de Dios, Padre todopoderoso. Desde allí ha de venir a juzgar a vivos y muertos. Creo en el Espíritu Santo, la santa Iglesia católica, la comunión de los santos, el perdón de los pecados, la resurrección de la carne y la vida eterna. Amén.}

\stopParallel

\selectlanguage{latin}
Laus Deo Virginique Matri


\end{document}


\part{Maitines (Matutinum)}


\input mid-matins


\rubrics{Es recomendable finalizar la recitación pública del Pequeño Oficio con la correspondiente Antífona Mariana, página \pageref{marian-ants}.}





\part{Vísperas (Vesperae)}

\startParallel
\latin{Deus \redgrecross in adjutórium meum intende.}
\vern{Oh Dios \redgrecross , ven en mi auxilio.}

\latin{Dómine, ad adjuvándum me festína.}
\vern{Señor, date prisa en socorrerme.}

\latin{Glória Patri, et Fílio, et Spirítui Sancto. Sicut erat in princípio, et nunc et semper, et in saécula saeculórum. Amen.}
\vern{Gloria al Padre,  al Hijo y al Espíritu Santo. Como era en el principio, ahora y siempre, por los siglos de los siglos. Amen.}
\stopParallel

\input mid-vespers

\rubrics{Es recomendable finalizar la recitación pública del Pequeño Oficio con la correspondiente Antífona Mariana, página \pageref{marian-ants}.}





\part{Completas (Completorium)}


\input mid-compline

\rubrics{La Hora de las Completas siempre concluye con la Antífona Mariana correspondiente, página \pageref{marian-ants}.}

\chapter{Antífonas Marianas}

\label{marian-ants}
\input marianantiphon






\end{document}

