
\gresetinitiallines1

\rubrics{La correspondiente Antífona Mariana se canta siempre después del rezo de las Completas. Después de las otras horas, se recomienda igualmente finalizar la recitación pública del Pequeño Oficio de la Bienaventurada Virgen María con la Antífona Mariana pertinente.}

\section{Alma Redemptoris Mater (Tiempos de Adviento y de Navidad)}
\label{almaredem}
\rubrics{Tono solemne:}

\greannotation{Ant.}
\greannotation{5.}
\gregorioscore{almaredemptoris_solemn}

\rubrics{Tono Simple:}{}

\greannotation{5.}
\gregorioscore{almaredemptoris}

\begin{verse}
    \tr{Madre del Redentor, virgen fecunda, puerta del cielo siempre abierta,\\
        estrella del mar, ven a librar al pueblo que tropieza y quiere levantarse.\\
        Ante la admiración de cielo y tierra engendraste a tu santo Creador,\\
        y permaneces siempre virgen. Recibe el saludo del ángel Gabriel,\\
        Oh, by what joy which Gabriel brought to thee,\\
        y ten piedad de nosotros, pecadores.

    }
\end{verse}

\rubrics{Durante el Tiempo de Adviento:}

\startParallel
\versicle  Angelus Dómini nuntiávit Maríae.
\switchcolumn
\tr{El ángel del Señor anunció a María.}
\switchcolumn
\response Et concépit de Spíritu Sancto.
\switchcolumn
\tr{Y concibió del Espíritu Santo.}
\switchcolumn
{Orémus.}
\switchcolumn
\tr{Oremos.}
\switchcolumn
Grátiam tuam, quǽsumus Dómine, méntibus nostris infúnde : \dag \ ut qui, Angelo nuntiánte, Christi Fílii tui Incarnatiónem cognóvimus, * per passiónem ejus et crucem ad resurrectiónis glóriam perducámur.  Per éumdem Christum Dóminum nostrum. \response . Amen.
\switchcolumn
\tr{Derrama, Señor, tu gracia sobre nuestros corazones; y al reconocer, por el anuncio del ángel, la Encarnación de tu Hijo Jesucristo, conducidos por su Pasión y Cruz, lleguemos a la gloria de su Resurrección. Por Jesucristo Nuestro Señor. Amén.}
\switchcolumn
{\versicle  Divínum auxílium máneat semper nobíscum. \response . Amen.}
\switchcolumn
\tr{Que el auxilio divino permanezca siempre con nosotros. Amén.}
\stopParallel


\bigskip

\rubrics{Desde el 24 de diciembre al 1° de febrero:}

\startParallel
\versicle  Post partum Virgo invioláta permansísti.
\switchcolumn
\tr{Después del parto, oh Virgen, has permanecido inviolada.}
\switchcolumn
{\response . Dei Génitrix, intercéde pro nobis.}
\switchcolumn
\tr{Madre de Dios intercede por nosotros.}
\switchcolumn
{Orémus.}
\switchcolumn
\tr{Oremos.}
\switchcolumn
Deus, qui salutis aetérnae, beátae Maríae virginitáte fecúnda, humáno géneri prǽmia praestitísti : \dag \ tríbue, quǽsumus; ut ipsam pro nobis intercédere sentiámus, * per quam me\-rú\-i\-mus auctórem vitae suscípere, Dó\-mi\-num nostrum Jesum Christum Fílium tuum. \response . Amen
\switchcolumn
\tr{Oh Dios, que por la fecunda virginidad de la bienaventurada María, diste al género humano los tesoros de la salvación eterna; concédenos, te rogamos, que experimentemos en favor nuestro la intercesión de Aquella por quien merecimos recibir al Autor de la vida, nuestro Señor Jesucristo, tu Hijo. Amén.}
\switchcolumn
{\versicle  Divínum auxílium máneat semper nobíscum. \response . Amen.}
\switchcolumn
\tr{Que el auxilio divino permanezca siempre con nosotros. Amen.}
\stopParallel

%\startParallel
%\latin{\versicle  Divínum auxílium máneat semper nobíscum.}
%%\vern{Que el auxilio divino permanezca siempre con nosotros.}
%\latin{\response . Amen.}
%\vern{Amén.}
%\stopParallel

\section{Ave Regina Coelorum (Septuagésima y Tiempo de Cuaresma)}
\label{averegina}

\rubrics{Desde el 2 de febrero hasta el miércoles Santo:}

\rubrics{Tono solemne:}{}

\greannotation{Ant.}
\greannotation{6.}
\gregorioscore{averegina_solemn}

\rubrics{Tono simple:}{}

\greannotation{6.}
\gregorioscore{averegina}

\begin{verse}
    \tr{
        Salve, Reina de los cielos y Señora de los ángeles;\\
        salve raíz, salve puerta;
        que dio paso a nuestra luz.\\
        Alégrate, Virgen gloriosa, entre todas la más bella;

        salve, oh hermosa doncella,
        ruega a Cristo por nosotros.

    }
\end{verse}

\startParallel
\versicle  Dignáre me laudáre te, Virgo sacráta.
\switchcolumn
\tr{Permíteme cantar tus alabanzas, Virgen sagrada.}
\switchcolumn
{\response . Da mihi virtútem contra hostes tuos.}
\switchcolumn
\tr{Hazme fuerte contra tus enemigos.}
\switchcolumn
{Orémus}
\switchcolumn
\tr{Oremos.}
\switchcolumn
Concéde, miséricors Deus, fragilitáti nostrae\ prae\-sí\-dium : \dag \
ut qui sanctae\ Dei Genitrícis memóriam ágimus, *
intercessiónis ejus auxílio, a nostris iniqui\-tá\-tibus resurgámus.
Per eúmdem Christum Dóminum nostrum.
\response . Amen.
\switchcolumn
\tr{¡Oh Dios de misericordia!, concédenos tu ayuda, pues somos débiles, para que, al recordar a la santa Madre de Dios, seamos auxiliados por su intercesión y nos levantemos de nuestra maldad. Por el mismo Jesucristo nuestro Señor. Amén.}
\switchcolumn
{\versicle  Divínum auxílium máneat semper nobíscum. \response . Amen.}
\switchcolumn
\tr{Que el auxilio divino permanezca siempre con nosotros. Amen.}
\stopParallel

%\startParallel
%\latin{\versicle  Divínum auxílium máneat semper nobíscum.}
%\switchcolumn
%\tr{Que el auxilio divino permanezca siempre con nosotros.}
%\latin{\response . Amen.}
%%\vern{Amén.}
%\stopParallel


\section{Regina Coeli (Tiempo de Pascua)}
\label{reginacaeli}

\rubrics{Desde el Domingo de Resurrección hasta el sábado después de Pentecostés:}

\rubrics{Tono solemne:}{}

\greannotation{Ant.}
\greannotation{6.}
\gregorioscore{reginacaeli_solemn}

\rubrics{Tono simple:}{}

\greannotation{6.}
\gregorioscore{reginacaeli}

\begin{verse}
    \tr{
        Reina del cielo, alégrate, aleluya,\\
        porque el Señor, a quien has merecido llevar, aleluya,\\
        ha resucitado, según su palabra, aleluya.\\
        Ruega al Señor por nosotros, aleluya.\\
        Gózate y alégrate, Virgen María, aleluya.\\
        Porque ha resucitado el Señor verdaderamente, aleluya.

    }
\end{verse}

\startParallel
\versicle  Gaude et laetáre Virgo María, allelúia.
\switchcolumn
\tr{Gózate y alégrate, Virgen María, aleluya.}
\switchcolumn
{\response . Quia surréxit Dóminus vere, allelúia.}
\switchcolumn
\tr{Porque ha resucitado el Señor verdaderamente, aleluya.}
\switchcolumn
{Orémus.}
\switchcolumn
\tr{Oremos.}
\switchcolumn
Deus, qui per resurrectiónem Fílii tui Dómini
nostri Jesu Christi mundum laetificáre dignatus es : \dag \
praesta, quaesumus; ut per ejus Genitrícem
Vírginem Maríam, * perpétuae\ capíamus
gáudia vitae. Per eúmdem Christum Dóminum nostrum. \response . Amen.
\switchcolumn
\tr{Oh Dios, que has alegrado al mundo por la resurrección de tu Hijo, nuestro Señor Jesucristo!, concédenos, por la intercesión de la Virgen María, su Madre, llegar a las alegrías de la eternidad. Por el mismo Jesucristo nuestro Señor. Amén.}
\switchcolumn
{\versicle  Divínum auxílium máneat semper nobíscum. \response . Amen.}
\switchcolumn
\tr{Que el auxilio divino permanezca siempre con nosotros. Amén.}
\stopParallel

%\startParallel
%\latin{\versicle  Divínum auxílium máneat semper nobíscum.}
%\vern{Que el auxilio divino permanezca siempre con nosotros.}
%\latin{\response . Amen.}
%\vern{Amén.}
%\stopParallel


\section{Salve Regina (Durante el año)}
\label{salveregina}

\rubrics{Desde el Domingo de la Santísima Trinidad hasta el sábado antes del Primer Domingo de Adviento:}

\rubrics{Tono solemne:}{}

\greannotation{Ant.}
\greannotation{1.}
\gregorioscore{salveregina_solemn}

\rubrics{Tono simple:}{}

\greannotation{5.}
\gregorioscore{salveregina}

\begin{verse}
    \tr{
        Dios te salve, Reina y Madre de misericordia,\\
        vida, dulzura y esperanza nuestra.\\
        Dios te salve, a ti clamamos los desterrados hijos de Eva;\\
        a ti suspiramos, gimiendo y llorando,\\
        en este valle de lágrimas.\\
        Ea, pues, Señora, abogada nuestra,\\
        vuelve a nosotros esos tus ojos misericordiosos,\\
        y después de este destierro, muéstranos a Jesús,\\
        fruto bendito de tu vientre.\\
        Oh clemente, oh piados, oh dulce Virgen María.

    }
\end{verse}

\startParallel
\versicle  Ora pro nobis sancta Dei Génitrix.
\switchcolumn
\tr{Ruega por nosotros, Santa Madre de Dios}
\switchcolumn
\response . Ut digni efficiámur promissiónibus Christi.
\switchcolumn
\tr{Para que seamos dignos de alcanzar las promesas de Cristo.}
\switchcolumn
Orémus.
\switchcolumn
\tr{Oremos.}
\switchcolumn
Omnípotens sempitérne Deus, qui gloriósae Vírginis Matris Maríae
corpus et ánimam, ut dignum Fílii tui habitáculum éffici mererétur,
Spíritu Sancto cooperánte praeparásti : \dag \  da, ut cujus
commemoratióne laetámur, * ejus pia intercessióne ab
instántibus malis et a morte perpétua liberémur.
Per eúmdem Christum Dóminum nostrum. \response . Amen.
\switchcolumn
\tr{Dios todopoderoso y eterno, con la ayuda del Espíritu Santo, preparaste el cuerpo y el alma de María, la Virgen Madre, para ser digna morada de tu Hijo; al recordarla con alegría, líbranos, por su intercesión, de los males presentes y de la muerte eterna. Por el mismo Jesucristo nuestro Señor. Amén.}
\switchcolumn
\versicle  Divínum auxílium máneat semper nobíscum. \response . Amen.
\switchcolumn
\tr{Que el auxilio divino permanezca siempre con nosotros. Amen.}
\stopParallel


