
\chapter{Salmodia}

\section{Salmodia Fuera de los Tiempos de Adviento y Navidad}


\subsection{Primer Salmo: Salmo 109}

\gregorioscore{gabc/an--dum_esset_rex--solesmes}

\gresetinitiallines0
\gregorioscore{psalms/109-3a-eng}

\startParallel

\input psalms/109vv

\switchcolumn

\tr{%Oráculo del Señor a mi Señor: Siéntate a mi derecha, 

	Y haré de tus enemigos estrado de tus pies.
\\
	Desde Sión extenderá el Señor el poder de tu cetro: somete en la batalla a tus enemigos.
\\
	Eres príncipe desde el día de tu nacimiento, entre esplendores sagrados; yo mismo te engendré, como rocío, antes de la aurora.
\\
	El Señor lo ha jurado y no se arrepiente: Tú eres sacerdote eterno, según el rito de Melquisedec.
\\
	El Señor a tu derecha, el día de su ira, quebrantará a los reyes.
\\
	Dará sentencia contra los pueblos, amontonará cadáveres, quebrantará cráneos sobre la ancha tierra.
\\
	En su camino beberá del torrente, por eso levantará la cabeza.
\\
	Gloria al Padre\ldots}
\stopParallel

\gregorioscore{gabc/an--dum_esset_rex--final}


\subsection{Segundo Salmo: Salmo 112}

\gresetinitiallines1
\gregorioscore{gabc/an--laeva_ejus--solesmes}


\gresetinitiallines0
\gregorioscore{psalms/112-4A-eng}

\startParallel
\input psalms/112vv
\switchcolumn

\tr{%Alabad, siervos del Señor, alabad el nombre del Señor.

	Bendito sea el nombre del Señor, ahora y por siempre:
\\
	De la salida del sol hasta su ocaso, alabado sea el nombre del Señor.
\\
	El Señor se eleva sobre todos los pueblos, su gloria sobre los cielos.
\\
	¿Quién como el Señor, Dios nuestro, que se eleva en su trono y se abaja para mirar al cielo y a la tierra?
\\
	Levanta del polvo al desvalido, alza de la basura al pobre,
\\
	Para sentarlo con los príncipes, los príncipes de su pueblo;
\\
	A la estéril le da un puesto en la casa, como madre feliz de hijos.
\\
	Gloria al Padre\ldots}
\stopParallel

\gregorioscore{gabc/an--laeva_ejus--final}

\subsection{Tercer Salmo: Salmo 121}

\gresetinitiallines1
\gregorioscore{gabc/an--nigra_sum_sed_formosa--solesmes}

\gresetinitiallines0
\gregorioscore{psalms/121-3b-eng}

\startParallel
\input psalms/121vv
\switchcolumn
\tr{%¡Qué alegría cuando me dijeron: * «Vamos a la casa del Señor»!

	Ya están pisando nuestros pies * tus umbrales, Jerusalén.
\\
	Jerusalén está fundada * como ciudad bien compacta.
\\
	Allá suben las tribus, las tribus del Señor, * según la costumbre de Israel, a celebrar el nombre del Señor;
\\
	En ella están los tribunales de justicia, * en el palacio de David.
\\
	Desead la paz a Jerusalén: * «Vivan seguros los que te aman,
\\
	Haya paz dentro de tus muros, * seguridad en tus palacios».
\\
	Por mis hermanos y compañeros, * voy a decir: «La paz contigo».
\\
	Por la casa del Señor, nuestro Dios, * te deseo todo bien.
\\
	Gloria al Padre\ldots}
\stopParallel

\gregorioscore{gabc/an--nigra-sum--final}

\subsection{Cuarto Salmo: Salmo 126}

\gresetinitiallines1
\gregorioscore{gabc/an--jam_hiems_transiit--solesmes}


\gresetinitiallines0
\gregorioscore{psalms/126-8G-eng}

\begin{paracol}{2}
	\input psalms/126vv
	\switchcolumn
	\tr{%Si el Señor no construye la casa, en vano se cansan los albañiles;

		Si el Señor no guarda la ciudad, en vano vigilan los centinelas.
\\
		Es inútil que madruguéis, que veléis hasta muy tarde, que comáis el pan de vuestros sudores:
\\
		¡Dios lo da a sus amigos mientras duermen! La herencia que da el Señor son los hijos; su salario, el fruto del vientre:
\\
		Son saetas en mano de un guerrero los hijos de la juventud.
\\
		Dichoso el hombre que llena con ellas su aljaba: no quedará derrotado cuando litigue con su adversario en la plaza.
\\
		Gloria al Padre\ldots}
\end{paracol}

\gregorioscore{gabc/an--jam_hiems_transiit--solesmes}

\subsection{Quinto Salmo: Salmo 147}

\gresetinitiallines1

\gregorioscore{gabc/an--speciosa_facta_es--init}


\gresetinitiallines0

\gregorioscore{psalms/147-4A-eng}

\begin{paracol}{2}
	\input psalms/147vv
	\switchcolumn
	\tr{% Glorifica al Señor, Jerusalén; alaba a tu Dios, Sión:

		Que ha reforzado los cerrojos de tus puertas, y ha bendecido a tus hijos dentro de ti;
\\
		Ha puesto paz en tus fronteras, te sacia con flor de harina.
\\
		Él envía su mensaje a la tierra, y su palabra corre veloz;
\\
		Manda la nieve como lana, esparce la escarcha como ceniza;
\\
		Hace caer el hielo como migajas y con el frío congela las aguas;
\\
		Envía una orden, y se derriten; sopla su aliento, y corren.
\\
		Anuncia su palabra a Jacob, sus decretos y mandatos a Israel;
\\
		Con ninguna nación obró así, ni les dio a conocer sus mandatos.
\\
		Gloria al Padre\ldots}
\end{paracol}

\gregorioscore{gabc/an--speciosa_facta_es--solesmes}

\bigskip

Las Vísperas continúan en la página \pageref{little-chapter} con el Pequeño Capítulo.

\section{Salmodia para el Tiempo de Adviento}

\subsection{Primer Salmo - Salmo 109}

\gresetinitiallines1

\gregorioscore{gabc/an--missus_est--init}

\gresetinitiallines0
\gregorioscore{gabc/109-8G}

\startParallel

\input psalms/109-8G

\switchcolumn

\tr{%Oráculo del Señor a mi Señor: Siéntate a mi derecha, 

	Y haré de tus enemigos estrado de tus pies.
\\
	Desde Sión extenderá el Señor el poder de tu cetro: somete en la batalla a tus enemigos.
\\
	Eres príncipe desde el día de tu nacimiento, entre esplendores sagrados; yo mismo te engendré, como rocío, antes de la aurora.
\\
	El Señor lo ha jurado y no se arrepiente: Tú eres sacerdote eterno, según el rito de Melquisedec.
\\
	El Señor a tu derecha, el día de su ira, quebrantará a los reyes.
\\
	Dará sentencia contra los pueblos, amontonará cadáveres, quebrantará cráneos sobre la ancha tierra.
\\
	En su camino beberá del torrente, por eso levantará la cabeza.
\\
	Gloria al Padre\ldots}
\stopParallel

\gregorioscore{gabc/an--missus_est}

\subsection{Segundo Salmo - Salmo 112}

\gresetinitiallines1
\gregorioscore{gabc/an--ave_maria..._alleluia--solesmes_1961}

\gresetinitiallines0
\gregorioscore{gabc/112-1g}

\startParallel
\input psalms/112-1g
\switchcolumn

\tr{%Alabad, siervos del Señor, alabad el nombre del Señor.

	Bendito sea el nombre del Señor, ahora y por siempre:
\\
	De la salida del sol hasta su ocaso, alabado sea el nombre del Señor.
\\
	El Señor se eleva sobre todos los pueblos, su gloria sobre los cielos.
\\
	¿Quién como el Señor, Dios nuestro, que se eleva en su trono y se abaja para mirar al cielo y a la tierra?
\\
	Levanta del polvo al desvalido, alza de la basura al pobre,
\\
	Para sentarlo con los príncipes, los príncipes de su pueblo;
\\
	A la estéril le da un puesto en la casa, como madre feliz de hijos.
\\
	Gloria al Padre\ldots}
\stopParallel

\gregorioscore{gabc/an--ave_maria..._alleluia--solesmes_1961}

\subsection{Tercer Salmo - Salmo 121}

\gresetinitiallines1
\gregorioscore{gabc/an--ne_timeas--init}

\gresetinitiallines0
\gregorioscore{gabc/121-8G}

\startParallel
\input psalms/121-8G.tex
\switchcolumn
\tr{%¡Qué alegría cuando me dijeron: * «Vamos a la casa del Señor»!

	Ya están pisando nuestros pies * tus umbrales, Jerusalén.
\\
	Jerusalén está fundada * como ciudad bien compacta.
\\
	Allá suben las tribus, las tribus del Señor, * según la costumbre de Israel, a celebrar el nombre del Señor;
\\
	En ella están los tribunales de justicia, * en el palacio de David.
\\
	Desead la paz a Jerusalén: * «Vivan seguros los que te aman,
\\
	Haya paz dentro de tus muros, * seguridad en tus palacios».
\\
	Por mis hermanos y compañeros, * voy a decir: «La paz contigo».
\\
	Por la casa del Señor, nuestro Dios, * te deseo todo bien.
\\
	Gloria al Padre\ldots}
\stopParallel

\gregorioscore{gabc/an--ne_timeas--solesmes}

\subsection{Cuarto Salmo - Salmo 126}

\gresetinitiallines1
\gregorioscore{gabc/an--dabit_ei_dominus--init}

\gresetinitiallines0
\gregorioscore{gabc/126-1f}

\begin{paracol}{2}
\input psalms/126-1f
	\switchcolumn
	\tr{%Si el Señor no construye la casa, en vano se cansan los albañiles;

		Si el Señor no guarda la ciudad, en vano vigilan los centinelas.
\\
		Es inútil que madruguéis, que veléis hasta muy tarde, que comáis el pan de vuestros sudores:
\\
		¡Dios lo da a sus amigos mientras duermen! La herencia que da el Señor son los hijos; su salario, el fruto del vientre:
\\
		Son saetas en mano de un guerrero los hijos de la juventud.
\\
		Dichoso el hombre que llena con ellas su aljaba: no quedará derrotado cuando litigue con su adversario en la plaza.
\\
		Gloria al Padre\ldots}
\end{paracol}

\gregorioscore{gabc/an--dabit_ei_dominus--solesmes_1961}

\subsection{Quinto Salmo - Salmo 147}

\gresetinitiallines1
\gregorioscore{gabc/an--ecce_ancilla_domini--init}

\gresetinitiallines0
\gregorioscore{gabc/147-8c}

\begin{paracol}{2}
\input psalms/147-8c
	\switchcolumn
	\tr{% Glorifica al Señor, Jerusalén; alaba a tu Dios, Sión:

		Que ha reforzado los cerrojos de tus puertas, y ha bendecido a tus hijos dentro de ti;
\\
		Ha puesto paz en tus fronteras, te sacia con flor de harina.
\\
		Él envía su mensaje a la tierra, y su palabra corre veloz;
\\
		Manda la nieve como lana, esparce la escarcha como ceniza;
\\
		Hace caer el hielo como migajas y con el frío congela las aguas;
\\
		Envía una orden, y se derriten; sopla su aliento, y corren.
\\
		Anuncia su palabra a Jacob, sus decretos y mandatos a Israel;
\\
		Con ninguna nación obró así, ni les dio a conocer sus mandatos.
\\
		Gloria al Padre\ldots}
\end{paracol}

\gregorioscore{gabc/an--ecce_ancilla_domini--}

\bigskip

Las Vísperas continúan en la página \pageref{little-chapter} con el Pequeño Capítulo.



\section{Salmodia para el Tiempo de Navidad}

\subsection{Primer Salmo: Salmo 109}

\gresetinitiallines1
\gregorioscore{gabc/an--o_admirabile}

\gresetinitiallines0
\gregorioscore{gabc/109-6}

\startParallel
\input psalms/109-6

\switchcolumn

\tr{%Oráculo del Señor a mi Señor: Siéntate a mi derecha, 

	Y haré de tus enemigos estrado de tus pies.
\\
	Desde Sión extenderá el Señor el poder de tu cetro: somete en la batalla a tus enemigos.
\\
	Eres príncipe desde el día de tu nacimiento, entre esplendores sagrados; yo mismo te engendré, como rocío, antes de la aurora.
\\
	El Señor lo ha jurado y no se arrepiente: Tú eres sacerdote eterno, según el rito de Melquisedec.
\\
	El Señor a tu derecha, el día de su ira, quebrantará a los reyes.
\\
	Dará sentencia contra los pueblos, amontonará cadáveres, quebrantará cráneos sobre la ancha tierra.
\\
	En su camino beberá del torrente, por eso levantará la cabeza.
\\
	Gloria al Padre\ldots}
\stopParallel

\gregorioscore{gabc/an--o_admirabile}

\subsection{Segundo Salmo: Salmo 112}

\gresetinitiallines1
\gregorioscore{gabc/an--quando_natus_es--solesmes_1961}

\gresetinitiallines0
\gregorioscore{gabc/112-3a2}

\startParallel
\input psalms/112-3a2
\switchcolumn

\tr{%Alabad, siervos del Señor, alabad el nombre del Señor.

	Bendito sea el nombre del Señor, ahora y por siempre:
\\
	De la salida del sol hasta su ocaso, alabado sea el nombre del Señor.
\\
	El Señor se eleva sobre todos los pueblos, su gloria sobre los cielos.
\\
	¿Quién como el Señor, Dios nuestro, que se eleva en su trono y se abaja para mirar al cielo y a la tierra?
\\
	Levanta del polvo al desvalido, alza de la basura al pobre,
\\
	Para sentarlo con los príncipes, los príncipes de su pueblo;
\\
	A la estéril le da un puesto en la casa, como madre feliz de hijos.
\\
	Gloria al Padre\ldots}
\stopParallel

\gregorioscore{gabc/an--quando_natus_es--solesmes_1961}

\subsection{Tercer Salmo: Salmo 121}

\gresetinitiallines1
\gregorioscore{gabc/an--rubum_quem--solesmes}

\gresetinitiallines0
\gregorioscore{gabc/121-4E}

\startParallel
\input psalms/121-4E.tex
\switchcolumn
\tr{%¡Qué alegría cuando me dijeron: * «Vamos a la casa del Señor»!

	Ya están pisando nuestros pies * tus umbrales, Jerusalén.
\\
	Jerusalén está fundada * como ciudad bien compacta.
\\
	Allá suben las tribus, las tribus del Señor, * según la costumbre de Israel, a celebrar el nombre del Señor;
\\
	En ella están los tribunales de justicia, * en el palacio de David.
\\
	Desead la paz a Jerusalén: * «Vivan seguros los que te aman,
\\
	Haya paz dentro de tus muros, * seguridad en tus palacios».
\\
	Por mis hermanos y compañeros, * voy a decir: «La paz contigo».
\\
	Por la casa del Señor, nuestro Dios, * te deseo todo bien.
\\
	Gloria al Padre\ldots}
\stopParallel

\gregorioscore{gabc/an--rubum_quem--solesmes}

\subsection{Cuarto Salmo: Salmo 126}

\gresetinitiallines1
\gregorioscore{gabc/an--germinavit_radix--solesmes}

\gresetinitiallines0
\gregorioscore{gabc/126-1f}

\startParallel
\input psalms/126-1f
	\switchcolumn
	\tr{%Si el Señor no construye la casa, en vano se cansan los albañiles;

		Si el Señor no guarda la ciudad, en vano vigilan los centinelas.
\\
		Es inútil que madruguéis, que veléis hasta muy tarde, que comáis el pan de vuestros sudores:
\\
		¡Dios lo da a sus amigos mientras duermen! La herencia que da el Señor son los hijos; su salario, el fruto del vientre:
\\
		Son saetas en mano de un guerrero los hijos de la juventud.
\\
		Dichoso el hombre que llena con ellas su aljaba: no quedará derrotado cuando litigue con su adversario en la plaza.
\\
		Gloria al Padre\ldots}
\stopParallel
\gregorioscore{gabc/an--germinavit_radix--solesmes}

\subsection{Quinto Salmo: Salmo 147}

\gresetinitiallines1
\gregorioscore{gabc/an--ecce_maria_genuit--solesmes}

\gresetinitiallines0
\gregorioscore{gabc/147-2}

\startParallel
\input psalms/147-2
	\switchcolumn
	\tr{% Glorifica al Señor, Jerusalén; alaba a tu Dios, Sión:

		Que ha reforzado los cerrojos de tus puertas, y ha bendecido a tus hijos dentro de ti;
\\
		Ha puesto paz en tus fronteras, te sacia con flor de harina.
\\
		Él envía su mensaje a la tierra, y su palabra corre veloz;
\\
		Manda la nieve como lana, esparce la escarcha como ceniza;
\\
		Hace caer el hielo como migajas y con el frío congela las aguas;
\\
		Envía una orden, y se derriten; sopla su aliento, y corren.
\\
		Anuncia su palabra a Jacob, sus decretos y mandatos a Israel;
\\
		Con ninguna nación obró así, ni les dio a conocer sus mandatos.
\\
		Gloria al Padre\ldots}
\stopParallel

\gregorioscore{gabc/an--ecce_maria_genuit--solesmes}






\chapter{Pequeño Capítulo}
\label{little-chapter}

\subsection{Fuera del Tiempo de Adviento:}

\gresetinitiallines1
\gregorioscore{chapter/vespers}

\subsection{Durante el Tiempo de Adviento:}

\gregorioscore{chapter/vespers-advent}
%Egrediétur virga de radíce Iesse, et flos de radíce ejus ascéndet. Et requiéscet super eum Spíritus Dómini.
%R. Deo grátias.




\chapter{Himno}

\subsection{Melodía del Pequeño Oficio}

Melodía del Oficio Parvo (la más simple)

\gregorioscore{gabc/hy--ave_maris_stella_(officium_parvum)--vatican}


\rubrics{A continuación sigue el Versículo \pageref{vespers-versicle}}


%	Melodía Alternativa (Dominica)

%	\gregorioscore{gabc/hy--ave_maris_stella_(ii)--dominican}

%	La melodía más conocida

\subsection{Melodía Alternativa (Dominica)}

\gregorioscore{gabc/hy--ave_maris_stella_(off._bmv_in_sab._alt.)--solesmes_1960}




\rubrics{A continuación sigue el Versículo \pageref{vespers-versicle}}

\subsection{La Melodía más Conocida}

\gregorioscore{gabc/hy--ave_maris_stella--solesmes}

\setcounter{versecount}{0}
\startParallel

\latin{Ave maris stella,
	Dei Mater alma,
	Atque semper Virgo,
	Felix cæli porta.}
\vern{Salve, estrella del mar,
	Santa Madre de Dios,
	y siempre Virgen,
	puerta dichosa del cielo.}

\latin{Sumens illud Ave
	Gabriélis ore,
	Funda nos in pace,
	Mutans Hevæ nomen.}
\vern{Al escuchar el Ave
	de boca de Gabriel,
	asegúranos en la paz,
	cambiando el nombre de Eva.}

\latin{Solve vincla reis,
	Profer lumen cæcis,
	Mala nostra pelle,
	Bona cuncta posce.}
\vern{Desata las cadenas a los pecadores,
	procura a los ciegos la luz,
	ahuyenta nuestros males,
	y alcánzanos todo bien.}

\latin{Monstra te esse matrem,
	Sumat per te preces,
	Qui pro nobis natus,
	Tulit esse tuus.}
\vern{Muestra que eres nuestra Madre,
	y Aquel que por nosotros
	quiso ser Hijo tuyo, reciba,
	por tu mediación, nuestras súplicas.}

\latin{Virgo singuláris,
	Inter omnes mitis,
	Nos culpis solútos
	Mites fac et castos.}
\vern{¡Oh Virgen sin igual,
	más pura que todas!, haznos,
	libres ya de las culpas,
	mansos y puros.}

\latin{Vitam præsta puram,
	Iter para tutum,
	Ut vidéntes Iesum,
	Semper collætémur.}
\vern{Haz que sea casta nuestra vida,
	prepáranos un camino seguro,
	para que, viendo a Jesús,
	gocemos contigo eternamente.}

\latin{Sit laus Deo Patri,
	Summo Christo decus,
	Spirítui Sancto,
	Tribus honor unus.
	Amen.}
\vern{Alabanza sea dada a Dios Padre,
	gloria a Cristo Rey,
	y al Espíritu Santo,
	honor igual a los tres.
	Amén.}

\latin{\versicle Diffúsa est grátia in lábiis tuis.}
\vern{\versicle En tus labios se derrama la gracia.}

\latin{\response Proptérea benedíxit te Deus in ætérnum.}
\vern{\response El Señor te bendice eternamente.}

\stopParallel


\section{Versículo}
\label{vespers-versicle}

\setcounter{versecount}{0}
\startParallel

\latin{\versicle Diffúsa est grátia in lábiis tuis.}
\vern{\versicle En tus labios se derrama la gracia.}

\latin{\response Proptérea benedíxit te Deus in ætérnum.}
\vern{\response El Señor te bendice eternamente.}

\stopParallel



\chapter{Magnificat}
\section{Per Annum.}

%Edición del Monasterio de Solesmes:

\gregorioscore{gabc/an--beata_mater..._intercede--solesmes}

%Edición Vaticana:

%	\gregorioscore{gabc/an--beata_mater--vatican}

\rubrics{Versión solemne:}

\gresetinitiallines0
\gregorioscore{psalms/Magnificat-solemn2}

\input psalms/magnificat2solemn

\rubrics{Versión simple}

\gregorioscore{psalms/magnificat2simple}

\input psalms/magnificat2simple

\gresetinitiallines1
\gregorioscore{gabc/an--beata_mater..._intercede--solesmes}

\section{Paschaltide.}

\gresetinitiallines1
\gregorioscore{reginacaeliAlt}
\gresetinitiallines0
\gregorioscore{gabc/Magnificat-solemn1D2}
\input psalms/magnificat1D2solemn
\gresetinitiallines1
\gregorioscore{reginacaeliAlt}


\section{Tiempo de Adviento}


\gresetinitiallines1
\gregorioscore{gabc/an--spiritus_sanctus}

\gresetinitiallines0
\gregorioscore{gabc/Magnificat-8G}

\input psalms/Magnificat-8G


\gresetinitiallines1
\gregorioscore{gabc/an--spiritus_sanctus}


\section{Tiempo de Navidad}

\gresetinitiallines1
\gregorioscore{gabc/an--magnum_haereditatis--solesmes}

\gresetinitiallines0
\gregorioscore{psalms/Magnificat-2}

%\input psalms/Magnificat2
\input psalms/magnificat2solemn

\gresetinitiallines1
\gregorioscore{gabc/an--magnum_haereditatis--solesmes}



\chapter{Oratio}

\gresetinitiallines0
\gregorioscore{gabc/domine-exaudi}
\gresetinitiallines1


\section{Collecta Per Annum}


\setcounter{versecount}{0}
\startParallel
\latin{
	Concéde nos fámulos tuos, quǽsumus, Dómine Deus, perpétua mentis et córporis sanitáte gaudére: et, gloriósa beátæ Maríæ semper Vírginis intercessióne, a præsénti liberári tristítia, et ætérna pérfrui lætítia.
}
\vern{Te pedimos, Señor, que nosotros, tus siervos, gocemos siempre de salud de alma y cuerpo; y, por la intercesión gloriosa de Santa María, la Virgen, líbranos de las tristezas de este mundo, y concédenos las alegrías del cielo.}

\latin{
	Per Dóminum nostrum Jesum Christum, Fílium tuum: qui tecum vivit et regnat in unitáte Spíritus Sancti Deus, per ómnia sǽcula sæculórum.}
\vern{Por nuestro Señor Jesucristo, tu Hijo, que vive y reina contigo en la unidad del Espíritu Santo, Dios, por todos los siglos de los siglos.}


\latin{\response Amen.}
\vern{\response Amén.}

\stopParallel


\section{Colecta para Adviento}

\setcounter{versecount}{0}
\startParallel
\latin{
	Deus, qui de beátae Maríae Vírginis útero Verbum tuum, Angelo nuntiánte, carnem suscípere voluísti: praesta supplícibus tuis; ut, qui vere eam Genitrícem Dei crédimus, ejus apud te intercessiónibus adjuvémur.
}
\vern{Dios, que quisiste que a la palabra del Ángel se encarnase tu Verbo en el seno de la bienaventurada Virgen María: haz, te suplicamos, que cuantos creemos que es verdaderamente Madre de Dios, seamos ayudados delante de ti con su intercesión.
}

\latin{
	Per eúndem Dóminum nostrum Jesum Christum Fílium tuum, qui tecum vivit et regnat in unitáte Spíritus Sancti, Deus, per ómnia sǽcula sæculórum.
}
\vern{Por el mismo Señor Nuestro Jesucristo, tu Hijo, que vive y reina en la unidad del Espíritu Santo, Dios, por todos los siglos de los siglos.}

\latin{\response Amen.}
\vern{\response Amén.}

\stopParallel


\section{Colecta para Navidad}

\setcounter{versecount}{0}
\startParallel
\latin{
	Deus, qui salútis ætérnæ, beátæ Maríæ virginitáte fecúnda, humáno géneri prǽmia præstitísti: tríbue, quǽsumus; ut ipsam pro nobis intercédere sentiámus, per  merúimus auctórem vitæ suscípere, Dóminum nostrum Jesum Christum Fílium tuum:
}
\vern{Oh Dios, que por la maternidad virginal de santa María entregaste al género humano los tesoros de la salvación eterna!, concédenos la intercesión de la Madre de nuestro Redentor.
}

\latin{
	Qui tecum vivit et regnat in unitáte Spíritus Sancti, Deus, per ómnia sǽcula sæculórum.
}
\vern{
	Que vive y reina en unión del Espíritu Santo, Dios, por todos los siglos de los siglos.
}

\latin{\response Amen.}
\vern{\response Amén.}

\stopParallel


\section{Conclusio}


\setcounter{versecount}{1}
\startParallel

\latin{\versicle Dómine, exáudi oratiónem meam.}
\vern{\versicle Señor, escucha nuestra oración.}

\latin{\response Et clamor meus ad te véniat.}
\vern{\response Y llegue a ti nuestro clamor.}
\stopParallel

% V. Dómine, exáudi oratiónem meam.

% R. Et clamor meus ad te véniat.


\gresetinitiallines0
\gregorioscore{gabc/or--benedicamus_domino_(in_festis_beatae_mariae_virginis)--sandhofe}
\gresetinitiallines1
%	V. Benedicámus Dómino.

%	R. Deo grátias.

% V. Fidélium ánimæ per misericórdiam Dei requiéscant in pace.

% R. Amen.

\setcounter{versecount}{1}
\startParallel

\latin{\versicle Fidélium ánimæ per misericórdiam Dei requiéscant in pace.}
\vern{\versicle  Las almas de los fieles, por la misericordia de Dios, descansen en paz.}

\latin{\response Amen.}
\vern{\response Amén.}
\stopParallel


